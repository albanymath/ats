% !TeX program = xelatex
\documentclass{UAmathtalk}

\author{Francesco Fournier-Facio}
\address{ETH Zürich, Switzerland}
\urladdr{https://people.math.ethz.ch/~fournief/}
\title{Hopfian Wreath Products and\\ the Stable Finiteness Conjecture}
\date{Thursday, November 10, 2022}


\begin{document}

\maketitle

\begin{abstract}
\emph{Wreath products} appear often as a case study in questions related to residual finiteness, thanks to a beautiful and simple characterization of Gruenberg. A related property is the Hopf property: a group is \emph{Hopfian} if every self-epimorphism is an isomorphism.

Every finitely generated residually finite group is Hopfian, which motivates looking at the Hopf property for wreath products, in hope of a simple characterization analogous to Gruenberg's.

It turns out that this problem is infinitely harder than Gruenberg's, even when focusing on the following special case: if G is finitely generated abelian, and H is finitely generated Hopfian, is the wreath product~$G \wr H$ Hopfian? We will see that this question is equivalent to one of the most longstanding open problems in group theory: Kaplansky's \emph{stable finiteness conjecture}, which is strongly related to the zero-divisor and idempotent conjectures, to the existence of a non-sofic group, and to Gottschalk's surjunctivity conjecture.

This is joint work with Henry Bradford (Cambridge).
\end{abstract}

\end{document}
