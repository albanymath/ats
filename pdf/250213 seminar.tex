% !TeX program = xelatex
\documentclass{UAmathtalk}

\author{Carolyn Abbott}
\address{Brandeis University}
\urladdr{https://www.carolynrabbott.com}
\title{Morse Boundaries of CAT(0)~Cube~Complexes}
\date{Thursday, February 13, 2025}

\renewcommand*{\where}{SS-256}


\begin{document}

\maketitle

\begin{abstract}
The visual boundary of a hyperbolic space is a quasi-isometry invariant that has proven to be a very useful tool in geometric group theory. In particular, there is a well-defined notion of the visual boundary of a hyperbolic group. When one considers CAT(0) spaces, however, the situation is more complicated, because the visual boundary is not a quasi-isometry invariant. Instead, one can consider a certain subspace of the visual boundary, called the (sublinearly) Morse boundary. In this talk, I will describe a new topology on this boundary and use it to show that the Morse boundary with the restriction of the visual topology is a quasi-isometry invariant in the case of (nice) CAT(0) cube complexes. This result is in contrast to Cashen’s result that the Morse boundary with the visual topology is not a quasi-isometry invariant of CAT(0) spaces in general. This is joint work with Merlin Incerti-Medici.
\end{abstract}

\end{document}
