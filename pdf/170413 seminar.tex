% !TeX program = xelatex
\documentclass{UAmathtalk}

\author{Rostislav Devyatov}
\address{University of Ottawa, Canada}
\urladdr{http://science.uottawa.ca/mathstat/en/people/devyatov-rostislav}
\title{On Invariant Ideals Of Representation Rings Of Semisimple Groups}
\date{Thursday, April 13, 2017}

\usepackage{amsfonts}

\begin{document}

\maketitle

\begin{abstract}
The talk is based on my joint work with Sanghoon Baek and Kirill Zainoulline; see \href{https://arxiv.org/abs/1612.07278/}{\texttt{[arXiv:1612.07278]}}.

To any semisimple group $G$, one can associate its weight lattice $\Lambda$, the set of simple weights $\varpi_1,\ldots,\varpi_n$, and the Weyl group $W$ acting on $\Lambda$. One can consider the Laurent polynomial ring $\mathbb{Z}[\Lambda]$ (the monomial corresponding to $\lambda\in\Lambda$ will be denoted by $e^{\lambda}$) and the \emph{augmented orbit polynomials} $p_i=-|W\varpi_i|+\sum_{\lambda\in W\varpi_i}e^{\lambda}$. These polynomials generate an ideal $I\subset\mathbb{Z}[\Lambda]$.

One can also consider the character lattice of the maximal torus of $G$: $T^*\subseteq\Lambda$ and the corresponding Laurent polynomial subring $\mathbb{Z}[T^*]\subseteq\mathbb{Z}[\Lambda]$.

If certain conditions on $T^*$ and $\Lambda$ are satisfied, I will explain how one can find the intersection $I\cap\mathbb{Z}[T^*]$.
\end{abstract}

\end{document}
