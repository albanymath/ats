% !TeX program = xelatex
\documentclass{UAmathtalk}

\author{Cristian Lenart}
\urladdr{http://www.albany.edu/~lenart/}
\title{Schubert Calculus Beyond \(K\)-Theory}
\date{Thursday, November 10, 2016}

\begin{document}

\maketitle

\begin{abstract}
Modern Schubert calculus has been mostly concerned with the study of the cohomology and $K$-theory (including their equivariant and quantum generalizations) of flag manifolds. The basic results for other cohomology theories have only been obtained recently; additional complexity is due to the dependence of the geometrically defined classes on reduced words for the corresponding Weyl group elements. After this main theory was developed, the next step is to derive explicit formulas. I will describe my work with K.~Zainoulline and C.~Zhong in this direction, which focuses on torus equivariant hyperbolic cohomology (a stalk version of elliptic cohomology). First, we generalize certain formulas for the equivariant Schubert classes in cohomology and $K$-theory. We also construct a canonical basis using the Kazhdan-Lusztig basis of a certain Hecke algebra, and study some of its properties.
\end{abstract}

\end{document}
