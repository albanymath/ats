% !TeX program = xelatex
\documentclass{UAmathtalk}

\author{Yiqiang Li}
\address{University at Buffalo, SUNY}
\urladdr{http://www.nsm.buffalo.edu/~yiqiang/}
\title{Quiver Varieties and Symmetric Pairs}
\date{Thursday, March 1, 2018}

\usepackage{amssymb}


\begin{document}

\maketitle

\begin{abstract}
To an \emph{ADE} Dynkin diagram, one can attach a simply-laced complex simple Lie algebra, say~$\mathfrak{g}$, and a class of Nakajima's quiver varieties.
The latter provides a natural home for a geometric representation theory of the former.
If the algebra~$\mathfrak{g}$ is further equipped with an involution, it leads to a so-called symmetric pair~$(\mathfrak{g},\mathfrak{k})$, where $\mathfrak{k}$ is the fixed-point subalgebra under involution.
In this talk, I’ll present bridges at several levels between symmetric pairs and Nakajima varieties.
\end{abstract}

\end{document}
