% !TeX program = xelatex
\documentclass{UAmathtalk}

\author{Matt Hogancamp}
\address{Northeastern University}
\urladdr{https://mathscinet.ams.org/mathscinet/MRAuthorID/948897}
\title{Hecke Algebras, Symmetric Functions, and~Their Categorification}
\date{Thursday, February 27, 2020}


\begin{document}

\maketitle

\begin{abstract}
There is a well known character map from the symmetric group algebra to the ring of symmetric functions (sends an $n$-cycle to the $n$-th power sum symmetric function).  This character has a $q$-analogue, relating Hecke algebras and symmetric functions with a parameter $q$.  In this talk I will give a graphical interpretation of both of these constructions, and discuss recent work with Gorsky and Wedrich in which we ``categorify'' this character map.  Roughly speaking, the Hecke algebra gets upgraded to the monoidal category of Soergel bimodules, the ring of symmetric functions gets upgraded to the category of modules over an explicit wreath product algebra, and the character map is a functor called the ``derived horizontal trace''.  No background in Soergel bimodules or heavy category theory will be assumed.
\end{abstract}

\end{document}
