% !TeX program = xelatex
\documentclass{UAmathtalk}

\author{Abdalrazzaq Zalloum}
\address{Queen’s University, Canada}
\urladdr{https://mast.queensu.ca/~abdul/}
\title{Regular languages for contracting geodesics in groups}
\date{Thursday, March 5, 2020}


\begin{document}

\maketitle

\begin{abstract}
A celebrated theorem due to Cannon states that for any finite generating set $S$ of a hyperbolic group, the language consisting of all geodesics in $\textrm{Cay}(G,S)$ is a regular language. This theorem has many interesting consequences, for example, it implies that the growth function of a hyperbolic group must be a rational function, for any finite generating set $S$. I will speak about our theorem with Eike generalizing the above theorem to all hyperbolic-like geodesics of a finitely generated group. In other words, our theorem states that for any finitely generated group $G$ and any finite generating set $S$, the language consisting of all hyperbolic-like geodesics in $\textrm{Cay}(G,S)$ is a regular language. A geodesic $c$ is said to be $D$-hyperbolic-like if there exists some $D>0$ such that every ball disjoint from $c$, once projected to $c$, projects to a subset of diameter at most $D$ (think about the axis of $a$ in $F_2=\langle a,b\rangle$). Also, a group $G=\langle S\rangle$ is hyperbolic if and only if there exists some $D$ such that every geodesic is $D$-hyperbolic like, and hence, our theorem recovers Cannon's theorem. The talk will be entirely self contained and accessible to everyone.
\end{abstract}

\end{document}
