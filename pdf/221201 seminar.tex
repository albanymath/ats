% !TeX program = xelatex
\documentclass{UAmathtalk}

\author{Woojin Kim}
\address{Duke University}
\urladdr{https://wj-kim.com}
\title{Persistence Diagrams at the Crossroads of Algebra and Combinatorics}
\date{Thursday, December 1, 2022}

\renewcommand*{\abstractsize}{\normalsize}

\begin{document}

\maketitle

\begin{abstract}
Persistent Homology (PH) is a method used in Topological Data Analysis (TDA) to extract multiscale topological features from data. Via PH, the multiscale topological features of a given dataset are encoded into a persistence module (indexed by a totally ordered set) and in turn, summarized by a persistence diagram.

In order to extend PH so as to be able to study wider types of data (e.g.\ time-varying point clouds), variations of the indexing set of persistence modules must inevitably occur, leading for example to multiparameter persistence modules, i.e.\ persistence modules indexed by the $n$-dimensional grid. It is however not always evident how to define a notion of persistence diagram for such variants.

This talk will introduce a generalized notion of persistence diagram for such variants which arises through exploiting both the principle of inclusion and exclusion from combinatorics and the canonical map from the limit to the colimit of a diagram of vector spaces (these being notions from category theory). We also discuss (1) how the generalized persistence diagram subsumes some other well-known invariants of multiparameter persistence modules and (2) algorithmic considerations for computing the generalized persistence diagram.

This talk is based on joint work with Nate Clause, Tamal Dey, Facundo Mémoli, and Samantha Moore.
\end{abstract}

\end{document}
