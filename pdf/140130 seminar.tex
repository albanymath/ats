% !TeX program = xelatex
\documentclass{UAmathtalk}

\usepackage{amssymb}

\author{Amanda Beecher}
\address{Ramapo College}
\urladdr{http://www.ramapo.edu/tas/faculty/amanda-beecher/}
\title{A Combinatorial Approach to Describing a Free Resolution of a Multigraded Module}
\date{Thursday, January 30, 2014}

\begin{document}

\maketitle

\begin{abstract}
Let $R=\Bbbk [x_1, \ldots , x_k]$ with the standard $\mathbb{Z}^k$ grading and suppose $L$ is a multigraded module of $R$.
Explicitly describing a free resolution of $L$ using completely combinatorial methods is an open problem.
Although there has been significant progress on this problem when $M=R/I$ for a monomial ideal $I$, many of the construction methods from this (also open) case do not generalize in a natural way to the setting of an arbitrary multigraded module.
This talk discusses a construction approach in the more general setting from a purely combinatorial perspective.
The advantage of this approach is the identification of a basis for the free modules in the resolution so that we can define the maps of the resolution directly.
We discuss some advantages and limitations to this technique.
This talk is accessible to graduate students.
\end{abstract}

\end{document}
