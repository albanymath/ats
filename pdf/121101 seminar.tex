\documentclass{UAmathtalk}

\author{Alex Tchervev}
\title{Betti Degrees Posets of Monomial Ideals}
\date{Thursday, November 1, 2012}

\begin{document}

\maketitle

\begin{abstract}
The theory of monomial ideals provides an important gateway for interaction between topology, combinatorics, geometry, and symbolic computation. One of the main outstanding problems in the field, dating back to the 1960s, and an area of very active current research, is the understanding of the structure of the minimal free resolution of a monomial ideal~$I$. In~this talk we will discuss how to interpret the minimal free resolution of~$I$ as a functor~$F=F(I)$ from the small category associated with a certain poset~$B=B(I)$ (the Betti degrees poset) to the category of chain complexes over a field. Such functors form an abelian category in a natural way, and it turns out that the functor $F$ is a projective object in that category. Leveraging this observation, we show that the isomorphism class of the poset $B$ completely determines the structure of the minimal free resolution, in the sense that given any monomial ideal $J$ with Betti degrees poset isomorphic to~$B$, then the minimal free resolution of~$J$ can be obtained in a canonical manner via the so called co-end construction from the functor~$F(I)$. This is joint work with Marco Varisco.
\end{abstract}

\end{document}
