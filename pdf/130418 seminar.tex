\documentclass{UAmathtalk}

\author{Thomas Fiore}
\address{University of Michigan-Dearborn}
\urladdr{http://www-personal.umd.umich.edu/~tmfiore/}
\title{Waldhausen Additivity and Approximation in Quasicategorical $K$-Theory}
\date{Thursday, April 18, 2013}

\addtolength{\textheight}{1ex}

\begin{document}

\maketitle

\begin{abstract}
Quasicategories (a.k.a.\ infinity categories) are simultaneously like topological spaces and categories.  Their theory, a beautiful blend of homotopy theory and category theory, has experienced rapid development and new applications in the past decade. In this talk, I will present two classical theorems of algebraic topology in this new context: Waldhausen Additivity and Approximation.  I will not assume any background on quasicategories for the talk. 

Waldhausen Additivity, in its most general form, says that Waldhausen $K$-theory sends a split-exact sequence $\mathcal{A} \rightarrow \mathcal{E} \rightarrow \mathcal{B}$ to a stable equivalence $K(\mathcal{E}) \rightarrow K(\mathcal{A}) \vee K(\mathcal{B})$ of spectra. I will sketch a proof for the case that $\mathcal{A}$, $\mathcal{E}$, and $\mathcal{B}$ are Waldhausen quasicategories satisfying mild hypotheses. This is joint work with Wolfgang L\"uck. The method here is to prove the classical theorem in an entirely simplicial way, combining elements of previous proofs, and then carry this proof over to quasicategories.

Waldhausen Approximation, on the other hand, provides one answer to the question: when does an equivalence of homotopy categories induce an equivalence of $K$-theory spectra?  In the quasicategorical context, it suffices for an exact functor to reflect cofibrations and induce an equivalence of homotopy categories. 
\end{abstract}

\end{document}
