% !TeX program = xelatex
\documentclass{UAmathtalk}

\author{Elden Elmanto}
\address{Harvard University}
\urladdr{https://eldenelmanto.com}
\title{Excision and Descent in\\ Motivic Cohomology Theories}
\date{Thursday, September 26, 2019}

\begin{document}

\maketitle

\begin{abstract}
The excision question in algebraic $K$-theory asks when certain bicartesian squares of (possibly noncommutative) rings are converted to bicartesian squares of $K$-theory spectra. This question has had computational ramifications via the resulting Mayer--Vietoris type long exact sequences. This question has been addressed by Suslin and Suslin--Wodzicki and a complete solution was elegantly offered by Land--Tamme. Their methods, however, were very ``noncommutative'' and do not address the excision question for cohomology theories which are only defined for commutative rings/schemes. Following Bhatt and Mathew, we offer a framework to address this problem and prove new excision results for certain cohomology theories represented in Morel--Voevodsky's category of motivic spectra. As a result we prove a motivic version of formal gluing à la Beauville--Laszlo. The main input is a new Grothendieck topology called the cdarc topology, whose sheaves can be controlled by cdh squares and their values on valuation rings.

This is joint work in progress with Hoyois, Iwasa, and Kelly.
\end{abstract}

\end{document}
