% !TeX program = xelatex
\documentclass{UAmathtalk}

\author{Kelly Spendlove}
\address{Rutgers University}
\urladdr{https://kelspendlove.github.io}
\title{Morse, Conley, and Computation}
\date{Thursday, April 11, 2019}

\begin{document}

\maketitle

\begin{abstract}
Algebraic topology and dynamical systems are intimately related: the algebra may constrain or force the existence of certain dynamics. Morse homology is the prototypical theory grounded in this observation. Conley theory is a far-reaching topological generalization of Morse theory. Within the Conley theory the connection matrix is the mathematical object which transforms the approach into a truly homological theory: it is the Conley-theoretic generalization of the Morse boundary operator.

We’ll discuss a new formulation of the connection matrix theory, which casts the connection matrix in categorical, homotopy-theoretic language.  This enables the efficient computation of connection matrices via the technique of reductions in combination with algebraic-discrete Morse theory. We will also discuss a software package for such computations and demonstrate the theory and computations with some applications to classical examples as well as a Morse theory on braids.
\end{abstract}

\end{document}
