\documentclass[12pt]{article}

\usepackage{fontspec,hyperref}
\setsansfont{Helvetica Neue}

\usepackage[empty]{fullpage}

\hypersetup{pdfinfo={
     Title={Algebra/Topology Seminar, September 20, 2012},
    Author={Marco Varisco},
   Subject={Andrew Linshaw, Invariant Subalgebras of Vertex Algebras},
  Keywords={seminar, announcement}
}}

\begin{document}

\noindent\hspace{-28pt}\raisebox{-19pt}{\XeTeXpicfile UAlogo.jpg scaled 340}%
\hfill\textsf{\textbf{\footnotesize\href{http://www.albany.edu/math/}{Department of Mathematics and Statistics}}}\bigskip\bigskip

\begin{center}\Large
  \textsf{\huge Algebra/Topology Seminar}\\[2.5\bigskipamount]
  \textsc{\LARGE Andrew Linshaw}\\
  Brandeis University\\[1.5\bigskipamount]
  \textsc{\LARGE Invariant Subalgebras of Vertex Algebras}\\[2\bigskipamount]
  Thursday, September 20, 2012\\1:15~p.m.\ in ES-143\\[3\bigskipamount]
\end{center}

\noindent\large\textsc{Abstract.}
Given a vertex algebra $A$ and a reductive group $G$ of automorphisms of~$A$, we discuss the question of when the invariant subalgebra~$A^G$ is strongly finitely generated.
This is an analogue of Hilbert's theorem on the finite generation of classical invariant rings.
We prove that this holds for any $G$ when $A$ is a free field algebra (either a bc system, $\beta\gamma$ system, or Heisenberg algebra), and when $A$ is the universal affine vertex algebra associated to a simple, finite-dimensional Lie algebra at generic level.

\end{document}
