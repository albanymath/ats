% !TeX program = xelatex
\documentclass{UAmathtalk}

\author{Amit Patel}
\address{Institute for Advanced Study, Princeton}
\urladdr{http://akpatel.org/}
\title{Reeb Spaces as Stratified Coverings}
\date{Thursday, March 26, 2015}

\DeclareMathAlphabet{\matheurm}{U}{eur}{m}{n}

\begin{document}

\maketitle

\begin{abstract}
The Reeb graph of a function tracks the connected components of its fibers.
If the function is stratifiable, then its Reeb graph is equivalent to a constructible cosheaf over the reals valued in~$\matheurm{Set}$.
For a map to a manifold~$M$, we may talk about its Reeb space.
If the map is stratifiable, then its Reeb space is equivalent to a constructible cosheaf over~$M$ valued in~$\matheurm{Set}$.

In this talk I will equate Reeb spaces, stratified coverings, and constructible cosheaves.
I will give a classification theorem for all three generalizing the classical classification theorem for ordinary coverings.
\end{abstract}

\end{document}
