% !TeX program = xelatex
\documentclass{UAmathtalk}

\author{Katharine Turner}
\address{Australian National University, Canberra, Australia}
\urladdr{https://sites.google.com/view/katharine-turner/}
\title{Decomposing the Persistent Homology Transform of Star-Shaped Objects}
\date{Thursday, October 17, 2024}

\renewcommand*{\where}{BB-B012}


\begin{document}

\maketitle

\begin{abstract}
In this talk, we study the geometric decomposition of the \mbox{degree-$0$} Persistent Homology Transform (PHT) as viewed as a persistence diagram bundle. We focus on star-shaped objects as they can be segmented into smaller, simpler regions known as “sectors”. Algebraically, we demonstrate that the degree-$0$ persistence diagram of a star-shaped object in~$\mathbb{R}^2$ can be derived from the degree-$0$ persistence diagrams of its sectors. Using this, we then establish sufficient conditions for star-shaped objects in~$\mathbb{R}^2$ so that they have “trivial geometric monodromy”. Consequently, the PHT of such a shape can be decomposed as a union of curves parameterized by~$S^1$, where the curves are given by the continuous movement of each point in the persistence diagrams that are parameterized by~$S^1$. Finally, we discuss the current challenges of generalizing these results to higher dimensions.
\end{abstract}

\end{document}
