% !TeX program = xelatex
\documentclass{UAmathtalk}

\author{Matt Zaremsky}
\urladdr{https://www.albany.edu/~mz498674/}
\title{Shift-Similar Groups of Permutations\\of the Natural Numbers}
\date{Thursday, September 15, 2022}

\renewcommand*{\when}{3:00 p.m.}


\begin{document}

\maketitle

\begin{abstract}
In joint work with Brendan Mallery (former UAlbany grad student, now at Tufts), we introduce the notion of a \emph{shift-similar} subgroup of the group of permutations of the natural numbers~$\mathbb{N}$. The definition makes use of the fact that any cofinite subset of~$\mathbb{N}$ is canonically bijective with~$\mathbb{N}$, and is an analog to the well-known condition of \emph{self-similarity} for subgroups of the group of automorphisms of a tree. In this talk I will discuss self-similarity and shift-similarity, compare and contrast them, and mention connections to the world of Thompson groups and Houghton groups. Perhaps the most striking result is that there exist uncountably many isomorphism classes of finitely generated shift-similar groups, unlike the self-similar case.
\end{abstract}

\end{document}
