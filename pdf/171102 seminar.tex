% !TeX program = xelatex
\documentclass{UAmathtalk}

\author{Justin Curry}
\urladdr{http://justinmcurry.com}
\title{Foundations of TDA:\\The Fiber of the Persistence Map}
\date{Thursday, November 2, 2017}

\usepackage{mathtools}

\begin{document}

\maketitle

\begin{abstract}
The persistence map is the map that sends a function on a topological space to its collection of persistence diagrams, which are canonical invariants of filtering a space by sublevel sets and taking homology in each degree. Geometrically, a persistence diagram is simply a configuration of points in the plane. In this talk I will study which configurations of points are possible and what the ramification of this map is for the simplest possible case---functions on the interval. Ongoing work and open problems will also be discussed.
\bigskip

\noindent N.B. This will be the first talk of a three-talk series on the foundations of topological data analysis (TDA) and will require minimal background knowledge.
\end{abstract}

\end{document}
