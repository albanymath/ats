% !TeX program = xelatex
\documentclass{UAmathtalk}

\author{Danika Van Niel}
\address{Binghamton University, SUNY}
\urladdr{https://www.danikavanniel.com}
\title{How Compatible Pairs of Transfer Systems Witness Equivariant Structure}
\date{Thursday, September 11, 2025}

\renewcommand*{\where}{Massry B010}


\begin{document}

\maketitle

\vspace{-\smallskipamount}
\begin{abstract}
Transfer systems are combinatorial objects that encode information about equivariant operations. More precisely, a transfer system encodes the transfers (or wrong-way maps) carried by algebras over certain equivariant operads. Thus, transfer systems allow us to use combinatorial tools to study equivariant homotopy theory. Compatible pairs of transfer systems, which are a pair of transfer systems satisfying certain conditions, correspond to multiplicative structures compatible with an underlying additive structure. In particular, compatible pairs are closely related to equivariant operads, ring spectra, bi-incomplete Tambara functors, and model structures. In this talk we introduce transfer systems, compatible pairs, and discuss several properties of transfer systems and their compatible pairs which help us better understand the equivariant objects and structures mentioned above.

Most of the work discussed in this talk began as a collaboration through the Women in Topology workshop and is joint with Kristen Mazur, Angélica M.\ Osorno, Constanze Roitzheim, Rekha Santhanam, and Valentina Zapata Castro. The other work discussed in this talk are from a collaboration with Sarah Klanderman, Chloe Lewis, Harlea Monson, and Koki Shibata, as well as a collaboration with David DeMark, Michael Hill, Yigal Kamel, Nelson Niu, Kurt Stoeckl, and Guoqi Yan.
\end{abstract}

\end{document}
