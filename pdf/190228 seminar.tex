% !TeX program = xelatex
\documentclass{UAmathtalk}

\author{Jeremy Hahn}
\address{MIT}
\urladdr{https://math.mit.edu/directory/profile.php?pid=2073}
\title{Even Spaces and Snaith Constructions}
\date{Thursday, February 28, 2019}

\usepackage{amsfonts}
\renewcommand*{\abstractsize}{\Large}

\begin{document}


\maketitle

\begin{abstract}
Call a CW-complex even if it has only even-dimensional cells and even-dimensional homotopy groups.  An example is the infinite complex projective space~$\mathbb{C}P^\infty$, which has only a single non-zero homotopy group in dimension~$2$.  I will survey work of Wilson that classifies all even spaces, as well as work of Hill and Hopkins that classifies certain group actions on even spaces.  I will then explain work of myself and Allen Yuan that extracts cohomology theories out of even spaces, the prototype of which is Snaith's construction of complex $K$-theory from~$\mathbb{C}P^\infty$.
\end{abstract}

\end{document}
