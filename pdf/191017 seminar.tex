% !TeX program = xelatex
\documentclass{UAmathtalk}

\author{Edgar Bering}
\address{Temple University}
\urladdr{https://math.temple.edu/~edgar.bering/}
\title{Special Covers of Alternating Links}
\date{Thursday, October 17, 2019}

\begin{document}

\maketitle

\begin{abstract}
The ``virtual conjectures'' in low-dimensional topology, stated by Thurston in~1982, postulated that every hyperbolic 3-manifold has finite covers that are Haken and fibered, with large Betti numbers. These conjectures were resolved in~2012 by Agol and Wise, using the machine of special cube complexes. Since that time, many mathematicians have asked how big a cover one needs to take to ensure one of these desired properties.

We begin to give a quantitative answer to this question, in the setting of alternating links in~$S^3$. If an alternating link~$L$ has a diagram with~$n$ crossings, we prove that the complement of~$L$ has a special cover of degree less than~$72((n-1)!)^2$. As a corollary, we bound the degree of the cover required to get Betti number at least~$k$. We also quantify residual finiteness, bounding the degree of a cover where a closed curve of length~$k$ fails to lift. This is joint work with David Futer.
\end{abstract}

\end{document}
