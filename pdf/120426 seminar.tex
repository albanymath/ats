\documentclass[12pt]{article}

\usepackage{fontspec,hyperref}
\setsansfont{Helvetica Neue}

\usepackage[empty]{fullpage}

\hypersetup{pdfinfo={
     Title={Algebra/Topology Seminar, April 26, 2012},
    Author={Marco Varisco},
   Subject={Arthur Lubovsky, A Combinatorial Model for Affine Crystals},
  Keywords={seminar, announcement}
}}

\begin{document}

\noindent\hspace{-28pt}\raisebox{-19pt}{\XeTeXpicfile UAlogo.jpg scaled 340}%
\hfill\textsf{\textbf{\footnotesize\href{http://www.albany.edu/math/}{Department of Mathematics and Statistics}}}\bigskip\bigskip

\begin{center}\Large
  \textsf{\huge Algebra/Topology Seminar}\\[2.5\bigskipamount]
  \textsc{Arthur Lubovsky}\\[\bigskipamount]
  \textsc{A Combinatorial Model for Affine Crystals}\\[2\bigskipamount]
  Thursday, April 26, 2012\\ 1:15~p.m.\ in ES-143
\end{center}\bigskip\bigskip

\noindent\large\textsc{Abstract.}
Crystals are colored directed graphs encoding information about Lie algebra representations. Crystals have various combinatorial models. C.~Lenart and A.~Postnikov defined the so-called alcove model for (highest weight) crystals. I will present a generalization, the quantum alcove model which is based on enumerating paths in the so-called quantum Bruhat graph of the corresponding finite Weyl group. The quantum alcove model is conjectured to model tensor products of Kirillov-Reshetikhin crystals of arbitrary Lie type. These are an important class of finite crystals (not of highest weight) for affine Lie algebras. There is reasonable evidence for this conjecture. For instance, it is proved for Lie types A and~C (i.e., for the special linear and symplectic algebras). The talk is based on joint work with C.~Lenart and is largely self-contained.

\end{document}
