% !TeX program = xelatex
\documentclass{UAmathtalk}

\author{Genevieve Walsh}
\address{Tufts University}
\urladdr{https://gwalsh01.pages.tufts.edu}
\title{Hyperbolic Groups, Relatively Hyperbolic Groups, and the Cannon Conjecture}
\date{Thursday, November 30, 2023}

\renewcommand*{\event}{Algebra/Topology Seminar --- Colloquium}%
\renewcommand*{\where}{BB-B012}


\begin{document}

\maketitle

\begin{abstract}
We describe several very interesting and rich classes of groups: hyperbolic groups and relatively hyperbolic groups. The fundamental group of a closed hyperbolic $3$-manifold is a good example of a hyperbolic group, and the fundamental group of a hyperbolic knot complement is a good example of a relatively hyperbolic group. We will explain some tools one can use with these groups, particularly their boundaries. We will explain a conjecture predicting when a hyperbolic group is a 3-manifold group. If time permits, we will discuss how filling and drilling is related to this conjecture.
\end{abstract}

\end{document}
