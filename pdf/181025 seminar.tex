% !TeX program = xelatex
\documentclass{UAmathtalk}

\author{Mike Lesnick}
\urladdr{https://www.albany.edu/~ml644186/}
\title{Computational Aspects of 2-Parameter Persistent Homology --- Part~1}
\date{Thursday, October 25, 2018}


\begin{document}

\maketitle

\begin{abstract}
In topological data analysis, one associates to the data a filtered topological space, whose structure we then examine using persistent homology. However, in many settings, a single filtered space is not a rich enough invariant to encode the interesting structure of the data. This motivates the study of multi-parameter persistence, which associates to the data a topological space simultaneously equipped with two or more filtrations. The homological invariants of these ``multi-filtered spaces,'' called persistence modules, are much richer than their 1-D counterparts, but also far more complicated. As such, adapting the usual 1-parameter persistent homology methodology for data analysis to the multi-parameter setting requires new ideas.

For the past several years, I have been working with \href{http://www.mlwright.org}{Matthew Wright} and several other collaborators on \href{http://rivet.online}{RIVET}, a practical software tool for the visualization and analysis of 2-parameter persistent homology. One key new feature of \href{http://rivet.online}{RIVET} is a fast (cubic time) algorithm for computing the minimal presentation of a 2-parameter persistence module. Perhaps surprisingly, the computational cost of the algorithm is similar in practice to that of the standard algorithm for computing 1-parameter persistent homology.

In this series of two talks, I'll introduce 2-parameter persistent homology, \href{http://rivet.online}{RIVET}, and our algorithm for computing minimal presentations.
\end{abstract}

\end{document}
