% !TeX program = xelatex
\documentclass{UAmathtalk}

\author{Jakob Hansen}
\address{University of Pennsylvania}
\urladdr{https://www.math.upenn.edu/~jhansen/}
\title{Cellular Sheaves, Discrete Hodge Theory, and Applications}
\date{Thursday, February 13, 2020}


\begin{document}

\maketitle

\begin{abstract}
Cellular sheaves are a discrete and computable representation of constructible sheaves over cell complexes, and naturally model many situations where data of varying types is parameterized by a space. Assigning inner products to the stalks of a cellular sheaf allows us to apply the constructions of discrete Hodge theory to its cochain complex. We obtain generalizations of the discrete Hodge Laplacians, including the ubiquitous graph Laplacian. These operators have interesting algebraic and spectral properties, and allow us to use the local-to-global structure of cellular sheaves in real-world systems where robustness is essential. This talk will outline the construction of cellular sheaves and their Laplacians, discuss the generalization of spectral graph theory to spectral sheaf theory, and sketch avenues for the application of sheaf Laplacians to realistic engineering and scientific problems.
\end{abstract}

\end{document}
