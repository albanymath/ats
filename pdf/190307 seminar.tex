% !TeX program = xelatex
\documentclass{UAmathtalk}

\author{Ren\'e Corbet}
\address{Graz University of Technology, Austria}
\urladdr{http://www.geometrie.tugraz.at/corbet/}
\title{A Kernel for Multi-Parameter Persistence and its Computation}
\date{Thursday, March 7, 2019}

\renewcommand*{\abstractsize}{\Large}

\begin{document}


\maketitle

\begin{abstract}
Topological data analysis and its main method, persistent homology, provide a toolkit for computing topological information of high-dimensional and noisy data sets. Kernels for one-parameter persistent homology have been established to connect persistent homology with machine learning techniques. In this talk, we discuss a kernel construction for multi-parameter persistence and why this kernel can provably be useful in applications. This is joint work with U.~Fugacci, M.~Kerber, C.~Landi, and B.~Wang.
\end{abstract}

\end{document}
