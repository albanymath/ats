\documentclass[12pt]{article}

\usepackage{fontspec,hyperref}
\setsansfont{Helvetica Neue}

\usepackage[empty]{fullpage}

\hypersetup{pdfinfo={
     Title={Algebra/Topology Seminar, August 30, 2012},
    Author={Marco Varisco},
   Subject={Boris Goldfarb, What is Applied Topology? The Case of Persistent Homology},
  Keywords={seminar, announcement}
}}

\begin{document}

\noindent\hspace{-28pt}\raisebox{-19pt}{\XeTeXpicfile UAlogo.jpg scaled 340}%
\hfill\textsf{\textbf{\footnotesize\href{http://www.albany.edu/math/}{Department of Mathematics and Statistics}}}\bigskip\bigskip

\begin{center}\Large
  \textsf{\huge Algebra/Topology Seminar}\\[2.5\bigskipamount]
  \textsc{\LARGE Boris Goldfarb}\\[1.5\bigskipamount]
  \textsc{\LARGE What is Applied Topology?}\\
  \textsc{The Case of Persistent Homology}\\[2\bigskipamount]
  Thursday, August 30, 2012\\1:15~p.m.\ in ES-143\\[3\bigskipamount]
\end{center}

\noindent\large\textsc{Abstract.}
How do you sort through data?  With such modern developments as the WWW, sensor networks, and digital instruments that allow collection of enormous amounts of data, how does one organize it?  How does one draw global observations from the individual data points?  A new popular technique is to compute the ``persistent homology'' of the data set.  I will explain what that means.

\end{document}
