% !TeX program = xelatex
\documentclass{UAmathtalk}

\author{Michael Ben-Zvi}
\address{Tufts University}
\urladdr{https://sites.tufts.edu/benzvi/}
\title{Combination Theorems for Non-positively Curved Groups and Their Boundaries}
\date{Thursday, September 27, 2018}


\begin{document}

\maketitle

\begin{abstract}
A combination theorem is a result which says when two groups in a given class can be combined to form a new group in the same class. I will introduce some results, both classical and modern, pertaining to combination theorems for negatively curved and non-positively curved groups. Then, given these constructions, one would like to know what else can be said about the new groups. I will approach this from the setting of boundaries and describe a combination theorem for building non-positively curved groups whose boundaries are path connected.
\end{abstract}

\end{document}
