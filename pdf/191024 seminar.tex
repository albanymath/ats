% !TeX program = xelatex
\documentclass{UAmathtalk}

\author{Greg Malen}
\address{Duke University}
\urladdr{https://gregmalen.com}
\title{The Topology and Structure\\ of Crystallized Polyforms}
\date{Thursday, October 24, 2019}

\begin{document}

\maketitle

\begin{abstract}
In this talk I will discuss the extremal topological problem of finding the minimal number of tiles required to support a specified number of holes in a polyform. A polyform is a planar shape constructed by gluing together a finite number of congruent polygons along their edges, and when it achieves the minimal number of tiles for its given number of holes, we say it is crystallized. In particular, I answer this question completely for any number of holes when the polygons are either equilateral triangles or squares, and I examine various structural conditions and obstructions to crystallization, and their impact on the enumeration of crystals. In the language of cell complexes, polyforms are a subset of strongly connected, pure 2-dimensional complexes, and this question is aligned with the more general study of maximizing codimension-1 homology. This is joint work with Érika Roldán at Ohio State.
\end{abstract}

\end{document}
