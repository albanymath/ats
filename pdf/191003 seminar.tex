% !TeX program = xelatex
\documentclass{UAmathtalk}

\author{Rylee Lyman}
\address{Tufts University}
\urladdr{https://sites.tufts.edu/lyman/}
\title{The Outer Automorphism Group of\\ a Free Product of Finite Groups}
\date{Thursday, October 3, 2019}

\usepackage{amsfonts}

\begin{document}

\maketitle

\begin{abstract}
Mapping class groups, $GL(n,\mathbb{Z})$, and~$Out(F_n)$, the outer automorphism group of a free group, are among some of the most well-studied infinite discrete groups. One facet they have in common is that, although finitely presented, they are ``big'' groups, in the sense that their elements exhibit a rich and wide array of dynamical behavior. The Nielsen--Thurston normal form, Jordan normal form, and relative train track representative, respectively, all attempt to expose and present this information in an organized way to aid reasoning about this behavior.

The group of outer automorphisms of a finite free product of finite groups is closely related to~$Out(F_n)$, but is comparatively understudied. In this talk we will introduce these groups, related geometric structures they act on, and review some of the known results. We would like to argue that these groups are also ``big'': to this end we have shown how to extend work of Bestvina, Feighn, and Handel to construct relative train track representatives for outer automorphisms of free products.
\end{abstract}

\end{document}
