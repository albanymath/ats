\documentclass[12pt]{article}

\usepackage[empty]{fullpage}
\usepackage{graphicx,helvet}

\begin{document}

\noindent\hspace{-28px}\raisebox{-19px}{%
\includegraphics[scale=0.34]{UAlogo.jpg}}%
\hfill\textsf{\textbf{\footnotesize%
Department of Mathematics and Statistics}}\bigskip\bigskip

\begin{center}\Large
  \textsf{{\huge Algebra/Topology Seminar}}\\[2.5\bigskipamount]
  \textsc{Mahdi Asgari}\\
  {\large Oklahoma State University and Cornell University}\\[\bigskipamount]
  \textsc{Counting Cusp Forms}\\[2\bigskipamount]
  Thursday, February 16, 2012\\ 1:15~p.m.\ in ES-143\\
  (tea \&\ coffee at 12:45 p.m.\ in ES-152)
\end{center}\bigskip\bigskip

\large\noindent\textsc{Abstract.}
Cusp forms on a (reductive or semisimple) algebraic group, such
as~$SL(n)$, are very basic objects in Number Theory. Until a few years ago
it was not known that there are infinitely many cusp forms on a group such
as~$SL(n)$ beyond very small values of~$n$. One way to count cusp forms is in
terms of their associated eigenvalue with respect to certain invariant
differential operators (such as the Laplacian) on the corresponding
locally symmetric space. Weyl's law refers to an asymptotic formula for
the number of cusp forms on a given connected reductive group, in
particular establishing their infinitude.

I will discuss some work-in-progress, joint with Werner M\"uller of
University of Bonn, establishing Weyl's law with remainder terms for
classical groups. Without remainder terms, Weyl's law was recently
established by Lindenstrauss and Venkatesh in a rather general setting.

\end{document}
