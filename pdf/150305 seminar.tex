% !TeX program = xelatex
\documentclass{UAmathtalk}

\author{An Huang}
\address{Harvard University}
\urladdr{http://www.math.harvard.edu/~anhuang/}
\title{Period Integrals and\\ Their Differential Systems}
\date{Thursday, March 5, 2015}

\begin{document}

\maketitle

\begin{abstract}
Period integrals are geometrical objects which can be realized as special functions, or sections of certain bundles.
Their origin goes back to Euler, Gauss, and Legendre in the study of complex algebraic curves.
In their modern version, period integrals naturally arise in Hodge theory, and more recently in mathematical physics, and the theory of hypergeometric functions.
I will give an overview of a recent program to use differential equations and D-module theory to study period integrals.
Connections to hypergeometric functions of Gel'fand-Kapranov-Zelevinsky (GKZ) will also be considered.
We will see that the theory is intimately related to a particular infinite dimensional representation of a reductive Lie algebra, and the topology of certain open varieties.
I will describe how the theory could help calculate period integrals, and offers new insights into the GKZ theory, and mirror symmetry.
This talk is based on joint works with S.~Bloch, B.~Lian, V.~Srinivas, S-T.~Yau, and X.~Zhu.
\end{abstract}

\end{document}
