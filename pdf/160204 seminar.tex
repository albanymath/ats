% !TeX program = xelatex
\documentclass{UAmathtalk}

\author{Jethro van Ekeren}
\address{Universidade Federal Fluminense, Rio de Janeiro, Brazil}
\urladdr{http://w3.impa.br/~jethro/}
\title{Classification of Simple Vertex Algebras in Rank~24}
\date{Thursday, February 4, 2016}

\begin{document}

\maketitle

\begin{abstract}
An interesting classical problem is the classification of integral lattices up to isomorphism. It turns out that self-dual (even, positive definite) integral lattices exist only for rank a multiple of~8, and that low rank examples are related to many beautiful, highly symmetrical objects---such as the~E8 lattice in rank~8, and the 24 Niemeier lattices in rank~24.

The problem for lattices generalizes very naturally to the classification of simple vertex algebras. I will speak about recent progress (joint work with N.~Scheithauer and S.~Moeller) on the classification of these algebras in rank~24, which are conjecturally 71 in number. More precisely, I will explain our determination of the fusion rings of fixed-point vertex algebras, and the classification of possible affine structures on simple vertex algebras in rank~24.
\end{abstract}

\end{document}
