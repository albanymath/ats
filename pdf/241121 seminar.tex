% !TeX program = xelatex
\documentclass{UAmathtalk}

\author{Teddy Einstein}
\address{Swarthmore College}
\urladdr{https://einstein.domains.swarthmore.edu/}
\title{Constructing Relatively Geometric Cubulations}
\date{Thursday, November 21, 2024}

\renewcommand*{\where}{BB-B012}


\begin{document}

\maketitle

\begin{abstract}
The study of hyperbolic and relatively hyperbolic groups acting properly and cocompactly on CAT(0) cube complexes has produced spectacular results in geometric group theory and low dimensional topology. In 2019, Groves and I introduced a new kind of action of a relatively hyperbolic group on a CAT(0) cube complex called a relatively geometric action. Relatively geometric actions are cocompact but not proper in a controlled way. In this talk, I will discuss some examples and applications of relatively geometric actions. I will also explain how to naturally construct relatively geometric actions using a relatively geometric version of Bergeron and Wise's boundary cubulation criterion for hyperbolic groups. This talk contains joint work with Daniel Groves, Suraj Krishna MS and Thomas Ng.
\end{abstract}

\end{document}
