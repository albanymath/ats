% !TeX program = xelatex
\documentclass{UAmathtalk}

\usepackage{amssymb}

\author{Michael Andrews}
\address{MIT}
\urladdr{https://math.mit.edu/people/profile?pid=1259}
\title{The Music of Spheres:\\ The Adams Spectral Sequence\\ and Periodicity}
\date{Thursday, April 24, 2014}

\begin{document}

\maketitle

\begin{abstract}
We'll begin with the notion of homotopy groups and see that they stabilise for spheres.
The talk will be concerned with a tool for computing these stable groups.
First, we'll examine why $Ext$ groups show up in topology.
After discussing the purpose of a spectral sequence we'll play with the Adams spectral sequence.
We'll get a feel for how algebraic relations are displayed by spectral sequence charts, how to read off homotopy groups from the charts, and we'll make precise to what extent the algebra is reflecting the topology. 
Then we'll take a step back to see repeating patterns in the charts and describe, to some extent, the topology underpinning these observations.
Finally, we'll see a theorem that completely describes the Adams spectral sequence at an odd prime~$p$ above a line of slope~$1/(p^2-p-1)$.
\end{abstract}

\end{document}
