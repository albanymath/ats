% !TeX program = xelatex
\documentclass{UAmathtalk}

\author{Nicholas Scoville}
\address{Ursinus College}
\urladdr{https://webpages.ursinus.edu/nscoville/}
\title{Towards a new digital homotopy theory}
\date{Thursday, February 20, 2020}


\begin{document}

\maketitle

\begin{abstract}
We present recent progress with collaborators Greg Lupton and John Oprea towards developing a digital version of homotopy theory.  An n-dimensional digital image is a finite subset of the integer lattice along with an adjacency relation.  Although there are many papers on digital homotopy theory,  many of the notions do not seem satisfactory from a homotopy point of view.  Indeed, some of the constructs most useful in homotopy theory, such as cofibrations and path spaces, are absent from the literature or completely trivial.  Working in the digital setting, we develop some basic ideas of homotopy theory, including cofibrations and path fibrations, in a way that seems more suited to homotopy theory.  We will also discuss recent results on the digital fundamental group, including a digital version of the Seifert-van Kampen theorem.  This talk will introduce some of the basics of digital topology and will not require any specialized background.
\end{abstract}

\end{document}
