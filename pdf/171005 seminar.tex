% !TeX program = xelatex
\documentclass{UAmathtalk}

\author{Cary Malkiewich}
\address{Binghamton University, SUNY}
\urladdr{http://www.math.uiuc.edu/~cmalkiew/}
\title{Periodic Orbits and Topological Restriction Homology}
\date{Thursday, October 5, 2017}

\begin{document}

\maketitle

\begin{abstract}
I will talk about a project to import trace methods, usually reserved for algebraic $K$-theory computations, into the study of periodic orbits of continuous dynamical systems (and vice-versa). Our main result so far is that a certain fixed-point invariant built using equivariant spectra can be ``unwound'' into a more classical invariant that detects periodic orbits. As a simple consequence, periodic-point problems (i.e., finding a homotopy of a continuous map that removes its $n$-periodic orbits) can be reduced to equivariant fixed-point problems. This answers a conjecture of Klein and Williams, and allows us to interpret their invariant as a class in topological restriction homology ($TR$), coinciding with a class defined earlier in the thesis of Iwashita and separately by Lück. This is joint work with Kate Ponto.
\end{abstract}

\end{document}
