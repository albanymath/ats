% !TeX program = xelatex
\documentclass{UAmathtalk}

\author{Jonathan Campbell}
\address{Vanderbilt University}
\urladdr{https://jonathanacampbell.com}
\title{Fixed Point Theory and\\ the Cyclotomic Trace}
\date{Thursday, September 20, 2018}


\begin{document}

\maketitle

\begin{abstract}
Fixed point theory has been the motivation for many of the most celebrated results of 20th century mathematics: the Lefschetz fixed point theorem, the Atiyah-Singer index theorem, and the development of étale cohomology. In this talk I'll describe work, joint with Kate Ponto, that relates classical fixed point theory to algebraic $K$-theory, topological Hochschild homology~($\mathit{THH}\,$), and the cyclotomic trace. The relationship seems to clarify both domains, and readily suggests generalizations, via machinery of Lindenstrauss-McCarthy, that relate to dynamical zeta functions. The link turns on careful considerations of the bicategorical structure of~$\mathit{THH}$.
\end{abstract}

\end{document}
