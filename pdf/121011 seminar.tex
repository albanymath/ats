\documentclass[12pt]{article}

\usepackage{fontspec,hyperref}
\setsansfont{Helvetica Neue}

\usepackage[empty]{fullpage}

\hypersetup{pdfinfo={
     Title={Algebra/Topology Seminar, October 11, 2012},
    Author={Marco Varisco},
   Subject={Fernando Guzman, Natural Duality for the Algebra of Conditional Logic},
  Keywords={seminar, announcement}
}}

\begin{document}

\noindent\hspace{-28pt}\raisebox{-19pt}{\XeTeXpicfile UAlogo.jpg scaled 340}%
\hfill\textsf{\textbf{\footnotesize\href{http://www.albany.edu/math/}{Department of Mathematics and Statistics}}}\bigskip\bigskip

\begin{center}\Large
  \textsf{\huge \href{http://www.albany.edu/~mv312143/seminar/}{Algebra/Topology Seminar}}\\[2.5\bigskipamount]
  \textsc{\LARGE Fernando Guzman}\\
  Binghamton University, SUNY\\[1.5\bigskipamount]
  \textsc{\LARGE Natural Duality for the Algebra\\ of Conditional Logic}\\[2\bigskipamount]
  Thursday, October 11, 2012\\1:15~p.m.\ in ES-143\\[3\bigskipamount]
\end{center}

\noindent\large\textsc{Abstract.}
This is joint work with Gina Kucinski.
The algebra of conditional logic, first studied by Guzman and Squier, is a model for short-circuit evaluation in programming languages like~$C$.
Here we will present a natural duality, in the sense of Clark and Davey, for the variety of $C$-algebras.
The dual category consists of partially ordered sets with bottom element, and a partially defined \textup{glb} operation.
These posets satisfy the property that existence of an upper bound implies existence of a greatest lower bound.

\end{document}
