% !TeX program = xelatex
\documentclass{UAmathtalk}

\author{Justin Curry}
\urladdr{http://justinmcurry.com}
\title{Combinatorial Problems from\\ Persistent Homology}
\date{Thursday, March 14, 2019}

\usepackage{amsfonts}

\begin{document}


\maketitle

\begin{abstract}
In this talk I will describe some combinatorics problems that come from studying inverse problems in persistence. First I'll review how many functions on the interval have the same persistence, when functions are identified up to orientation-preserving homeomorphisms of the domain. For two-dimensional base spaces, I will consider functions that factor as embeddings into~$\mathbb{R}^3$ followed by projection onto the $z$-axis. Functions are considered to be ``height equivalent'' if they are related by level-set preserving isotopy. A lower bound on the number of height equivalence classes is easily proved, but some work in progress on a conjectured upper bound will also be presented.
\end{abstract}

\end{document}
