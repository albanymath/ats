% !TeX program = xelatex
\documentclass{UAmathtalk}

\author{Cristian Lenart}
\urladdr{http://www.albany.edu/~lenart/}
\title{Macdonald Polynomials, Quantum \texorpdfstring{\(K\)}{K}-Theory, and Affine Lie Algebra Representations}
\date{Thursday, March 30, 2017}

\begin{document}

\maketitle

\begin{abstract}
The (symmetric) Macdonald polynomials are Weyl group invariant polynomials with rational function coefficients in $q,t$, which specialize to the irreducible characters of semisimple Lie algebras upon setting $q=t=0$. Quantum $K$-theory is a $K$-theoretic generalization of quantum cohomology. Kirillov-Reshetikhin (KR) modules are certain finite-dimensional modules for affine Lie algebras. Braverman and Finkelberg related the Macdonald polynomials specialized at $t=0$ to the quantum $K$-theory of flag varieties.  With S.~Naito, D.~Sagaki, A.~Schilling, and M.~Shimozono, I proved that the same specialization of Macdonald polynomials equals the graded character of a tensor product of (one-column) KR modules. I will discuss the combinatorics underlying both of these connections. The talk is largely self-contained.
\end{abstract}

\end{document}
