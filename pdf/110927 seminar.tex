\documentclass[12pt]{article}

\usepackage[empty]{fullpage}
\usepackage{graphicx,helvet}

\begin{document}

\noindent\hspace{-28px}\raisebox{-19px}{%
\includegraphics[scale=0.34]{UAlogo.jpg}}%
\hfill\textsf{\textbf{\footnotesize%
Department of Mathematics and Statistics}}\bigskip\bigskip

\begin{center}\Large
  \textsf{{\huge Algebra/Topology Seminar}}\\[2.5\bigskipamount]
  \textsc{Emanuele Delucchi}\\
  {\large Bremen University, Germany}\\[\bigskipamount]
  \textsc{Toric Arrangements:\\ Combinatorial Models and the Fundamental Group}\\[2\bigskipamount]
  \underline{Tuesday}, September 27, 2011\\ 1:15~p.m.\ in ES-146\\
  (tea \&\ coffee at 12:45 a.m.\ in ES-152)
\end{center}\bigskip\bigskip

\large\noindent\textsc{Abstract.}
The study of arrangements of subtori in the complex torus~$T$ is a recently thriving topic.  It has some structural similarities with the theory of hyperplane arrangements, yet it bears its own peculiarities.

Recall that the Salvetti Complex is a combinatorial model of the complement of a complexified arrangement of hyperplanes.  We take Salvetti's work as a stepping stone to develop a combinatorial model for the complement~$M:=T\setminus A$, where $A$ is any toric arrangement.  More precisely, we prove that~$M$ is homotopy equivalent to the nerve of a combinatorially defined acyclic category. Then we find a presentation of the fundamental group~of~$M$.

This is joint work with Giacomo D'Antonio, Bremen University.

\end{document}
