% !TeX program = xelatex
\documentclass{UAmathtalk}

\author{Andras Lorincz}
\address{University of Connecticut}
\title{Bernstein-Sato Polynomials\\ for Semi-Invariants of Quivers}
\date{Thursday, October 23, 2014}

\begin{document}

\maketitle

\begin{abstract}
The Bernstein-Sato polynomial is a numerical invariant that has applications to singularity theory, monodromy theory, etc.
In this talk I will present a technique for the computation of Bernstein-Sato polynomials in an equivariant setting.
After giving a quick background on Bernstein-Sato polynomials, prehomogeneous vector spaces, and semi-invariants of quivers, I will describe by examples the calculation of the Bernstein-Sato polynomials for quivers of Dynkin and extended Dynkin type.
In particular, these computations reveal information about the geometry of some orbit closures.
\end{abstract}

\end{document}
