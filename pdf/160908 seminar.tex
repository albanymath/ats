% !TeX program = xelatex
\documentclass{UAmathtalk}

\author{Jiuzu Hong}
\address{University of North Carolina at Chapel Hill}
\urladdr{http://hong.web.unc.edu}
\title{The Verlinde Formula for the Trace\\ of the Diagram Automorphism\\[\smallskipamount] on Conformal Blocks}
\date{Thursday, September 8, 2016}

\begin{document}

\maketitle

\begin{abstract}
Roughly speaking the conformal blocks are the spaces of generalized theta functions on moduli stack of parabolic $G$-bundles on projective curves.
Their dimensions can be computed by Verlinde formula.
Given a diagram automorphism of a simply-laced simple algebraic group, it induces operators on conformal blocks.
I will explain an analogue of Verlinde formula for the trace of these operators.
A mysterious non simply-laced group appears in this formula.
It is related to Jantzen's twining formula for weight spaces of representations and the analogue of twining formula for tensor invariant spaces by Shen and myself.
\end{abstract}

\end{document}
