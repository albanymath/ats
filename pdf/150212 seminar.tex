% !TeX program = xelatex
\documentclass{UAmathtalk}

\author{Ela Celikbas}
\address{University of Connecticut}
\title{Decomposing Gorenstein Rings\\ as Connected Sums}
\date{Thursday, February 12, 2015}

\renewcommand*{\abstractsize}{\normalsize}

\begin{document}

\maketitle

\begin{abstract}
In topology, amalgamating two manifolds near a chosen point on each creates another manifold, called a \emph{``connected sum.''} This concept plays a significant role in the classification of closed surfaces. Connected sums in algebra are related to connected sums in topology through an expression of the cohomology algebras. In commutative ring theory, these objects have been studied under different names and from different perspectives by various authors starting with Sah~(1974) in the graded case, and by Lescot~(1982) in the local case. A topologically influenced version was also studied in recent work by Smith and Stong, and quite a few authors approach this area via Macaulay's inverse systems.

Gorenstein rings, due to their various kinds of symmetries and duality properties, form an important and ubiquitous class of rings. In 2012 Ananthnarayan, Avramov and Moore introduced a new construction of Gorenstein rings. They defined a \emph{connected sum} of two Gorenstein local rings as an appropriate quotient of their fiber product. Although the fiber product is rarely Gorenstein, they proved that a connected sum of two Gorenstein local rings is always a Gorenstein ring. 

In this talk we discuss connected sums $R\#_k S$ of Gorenstein Artin local rings $R$ and~$S$ over their common residue field~$k$. We show that Gorenstein Artin local algebras can be decomposed as connected sums if their associated graded rings satisfy a certain condition. This talk is based on a recent joint work with H.~Ananthnarayan and Z.~Yang.
\end{abstract}

\end{document}
