% !TeX program = xelatex
\documentclass{UAmathtalk}

\author{Marco Varisco}
\urladdr{http://www.albany.edu/~mv312143/}
\title{Assembly Maps for Topological Cyclic Homology of Group Algebras --- Part 3}
\date{Thursday, November 19, 2015}

\begin{document}

\maketitle

\begin{abstract}
This will be the last talk in the series, and it will be largely independent of the previous one.
I will first describe how topological cyclic homology ($TC$) and other related theories ($TR$, $C$, $\ldots$) are built from topological Hochschild homology ($THH$).
I will then review our analysis of the assembly maps for $THH$ of group algebras, and explain how---and under which additional assumptions on the group and on the coefficient ring or ringspectrum---this leads to injectivity and rational injectivity results for the assembly maps for $C$, $TR$, $TC$, $\ldots.$
I will also discuss the possible failure of \mbox{surjectivity.}
Some of these results are from our recent preprint  \href{http://www.arxiv.org/abs/1504.03674/}{\texttt{arXiv:1504.03674}} with Wolfgang Lück, Holger Reich, and John Rognes, and some are new and will appear in a forthcoming article.
I will highlight how they use earlier joint work with Reich on the Adams isomorphism for equivariant orthogonal spectra, and with Lück and Reich on commuting homotopy limits and smash products.
% In this series of talks I will present joint work with Wolfgang Lück, Holger Reich, and John Rognes on assembly maps for topological cyclic homology and related theories of group algebras.
% Topological cyclic homology was invented by Bökstedt, Hsiang, and Madsen in their celebrated work on the $K$-theoretic Novikov Conjecture, and provides a powerful tool for computations in algebraic $K$-theory.
% It is a subtle equivariant refinement of Bökstedt's topological Hochschild homology, which is a stable homotopy theoretic version of Hochshild homology, which in turn can be seen as the linear analog of Waldhausen's cyclic nerve.
% After reviewing the construction of all these theories, I will explain how to use assembly maps to study them in the case of group rings or ringspectra.
% I will then present our general isomorphism, injectivity, and rational injectivity results for these assembly maps, and discuss the possible failure of surjectivity.
% Some of these results are from our recent preprint \href{http://www.arxiv.org/abs/1504.03674/}{\texttt{arXiv:1504.03674}}, and some are new and will appear in a forthcoming article.
\end{abstract}

\end{document}
