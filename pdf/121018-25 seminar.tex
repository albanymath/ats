\documentclass[12pt]{article}

\usepackage{fontspec,hyperref}
\setsansfont{Helvetica Neue}

\usepackage[top=.66in, bottom=.5in, left=1in, right=1in]{geometry}

\hypersetup{pdfinfo={
     Title={Algebra/Topology Seminar, October 18 and 25, 2012},
    Author={Marco Varisco},
   Subject={Bill Dunbar, Diameters of Sphere Quotients},
  Keywords={seminar, announcement}
}}

\begin{document}

\thispagestyle{empty}

\noindent\hspace{-28pt}\raisebox{-19pt}{\XeTeXpicfile UAlogo.jpg scaled 340}%
\hfill\textsf{\textbf{\footnotesize\href{http://www.albany.edu/math/}{Department of Mathematics and Statistics}}}\bigskip

\begin{center}\Large
  \textsf{\huge \href{http://www.albany.edu/~mv312143/seminar/}{Algebra/Topology Seminar}}\\[1.5\bigskipamount]
  \textsc{\LARGE Bill Dunbar}\\
  Bard College at Simon's Rock\\[1\bigskipamount]
\end{center}
\noindent\hrulefill
\begin{center}\Large
  \textsc{\LARGE Diameters of Sphere Quotients: \Large Introduction}\\[.75\bigskipamount]
  Thursday, October 18, 2012, 1:15~p.m.\ in ES-143\\[1\bigskipamount]
\end{center}

\noindent\large\textsc{Abstract.}
As a warm-up for a later talk on 3-sphere quotients, I'll discuss the solution of the corresponding 2-dimensional questions: if you let a finite subgroup of~$O(3)$ act on the unit 2-sphere, and define a distance function on the quotient space by saying that the distance between two orbits is the minimum (spherical) distance between a point in one orbit and a point in the other orbit, then what is the diameter of the resulting metric space?  Which subgroup has minimum diameter?

When extending this analysis to the next dimension, it is helpful to take advantage of the fact that $SO(4)$ is ``almost'' a direct product of $SO(3)$ with itself.  I'll discuss the (algebraic) classification of subgroups of product groups in general, and then (time permitting) explain how to classify (up to conjugacy in~$O(4)$) first the finite subgroups of~$SO(4)$, and then the finite subgroups of~$O(4)$, following Threlfall \&~Seifert, Du Val, and later Conway \&~Smith.\bigskip

\noindent\hrulefill
\begin{center}\Large
  \textsc{\LARGE Diameters of 3-Sphere Quotients}\\[.75\bigskipamount]
  Thursday, October 25, 2012, 1:15~p.m.\ in ES-143\\[1\bigskipamount]
\end{center}

\noindent\large\textsc{Abstract.}
I will report on joint work with Sarah Greenwald, Jill McGowan and Catherine Searle, resulting in lower bounds for diameters of quotients of~$S^3$ by closed subgroups of~$O(4)$ which act non-transitively ($S^3$~denotes the unit 3-sphere).  My contribution was in the case where the subgroup is finite (so the quotient is a spherical orbifold of dimension three), but I will also discuss the other cases (when the orbit space has dimension one or two).  The punch line is that the diameter is at least~$\arccos\left(\tan(3\pi/10)/\sqrt{3}\,\right)\!/2$.

\end{document}
