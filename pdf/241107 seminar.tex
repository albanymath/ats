% !TeX program = xelatex
\documentclass{UAmathtalk}

\author{Danika Van Niel}
\address{Binghamton University, SUNY}
\urladdr{https://www.danikavanniel.com}
\title{$C_{p^r q^s}$ Compatible Transfer Systems}
\date{Thursday, November 7, 2024}

\renewcommand*{\where}{BB-B012}


\begin{document}

\maketitle

\begin{abstract}
Transfer systems are combinatorial objects that encode information about equivariant operations. More precisely, a transfer system encodes the transfers (or wrong-way maps) carried by algebras over certain equivariant operads. Thus, transfer systems allow us to use combinatorial tools to study equivariant homotopy theory. Compatible pairs of transfer systems, which are a pair of transfer systems satisfying certain conditions, correspond to multiplicative structures compatible with an underlying additive structure. In particular, compatible pairs are closely related to bi-incomplete Tambara functors. In this talk we introduce transfer systems, compatible pairs, and discuss when a transfer system is only compatible with at most two other transfer systems. The work discussed in this talk began as a collaboration through the Women in Topology workshop and is joint with Kristen Mazur, Angélica M. Osorno, Constanze Roitzheim, Rekha Santhanam, and Valentina Zapata Castro.
\end{abstract}

\end{document}
