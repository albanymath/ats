\documentclass[12pt]{article}

\usepackage[empty]{fullpage}
\usepackage{graphicx,helvet,amssymb}

\begin{document}

\noindent\hspace{-28px}\raisebox{-19px}{%
\includegraphics[scale=0.34]{UAlogo.jpg}}%
\hfill\textsf{\textbf{\footnotesize%
Department of Mathematics and Statistics}}\bigskip\bigskip

\begin{center}\Large
  \textsf{{\huge Algebra/Topology Seminar}}\\[2\bigskipamount]
  \textsc{Courtney Thatcher}\\
  {\large Bard College at Simon's Rock}\\[\bigskipamount]
  \textsc{Lens Spaces and Free $\mathbb{Z}/p\mathbb{Z}$ Actions\\ on Products of Spheres}\\[\bigskipamount]
  Thursday, December 1, 2011\\ 1:15~p.m.\ in ES-146\\
  (tea \&\ coffee at 12:45 p.m.\ in ES-152)
\end{center}\bigskip

\large\noindent\textsc{Abstract.}
A (fake) lens space is the orbit space of a free finite cyclic group action on an odd dimensional sphere.  Lens spaces provide interesting examples for classification techniques in homotopy theory and were one of the first great triumphs of the techniques of surgery theory.  The classification of lens spaces generally involves three pieces: the homotopy classification using $k$-invariants; the simple homotopy classification using Reidemeister torsion; and the homeomorphism classification using surgery theory.

In this talk we will discuss some of the background and techniques used in the lens space classification and then extend the question to the classification of free cyclic large prime order group actions on products of spheres.  In the product of spheres case, the equivariant homotopy type will be determined and the simple structure set discussed. Similar to lens spaces, the first $k$-invariant generally determines the homotopy type, however for homotopy equivalences between products of an even number of spheres the Whitehead torsion vanishes and the quotients are also simple homotopy equivalent. Unlike lens spaces which are determined by their Reidemeister torsion and rho-invariant, the rho-invariant vanishes for products of an even number of spheres and the Pontrjagin classes are $p$-localized homeomorphism invariants for a given dimension.

\end{document}
