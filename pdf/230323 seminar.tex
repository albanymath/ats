% !TeX program = xelatex
\documentclass{UAmathtalk}

\author{Florian Russold}
\address{Graz University of Technology, Austria}
\urladdr{http://www.geometrie.tugraz.at/russold/}
\title{Persistent (Co)Sheaves and Decorated~Mapper Graphs}
\date{Thursday, March 23, 2023}

\renewcommand*{\where}{BB-B008}


\begin{document}

\maketitle

\begin{abstract}
I will discuss a new tool to analyze the topological structure of data sets which we call decorated mapper graphs. Decorated mapper graphs generalize ordinary mapper graphs which have been successfully applied in many areas. A decorated mapper graph can be viewed as a discrete approximation of the Leray cosheaf. I will explain how we use the theory of persistent cosheaves and cosheaf homology to show that cellular approximations of the Leray cosheaf with respect to a sequence of covers converge to the Leray cosheaf if the resolution of the covers goes to zero.
\end{abstract}

\end{document}
