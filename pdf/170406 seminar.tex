% !TeX program = xelatex
\documentclass{UAmathtalk}

\author{Thomas Zaslavsky}
\address{Binghamton University, SUNY}
\urladdr{http://people.math.binghamton.edu/zaslav/}
\title{Nice Queens -- No Attacks}
\date{Thursday, April 6, 2017}

\begin{document}

\maketitle

\begin{abstract}
The $q$-Queens Problem, a generalization of the famous $n$-Queens Problem, asks for the number of ways to place $q$ chess queens on an $n$-by-$n$ chessboard so that none attacks another.  Seth Chaiken, Christoper Hanusa, and I have made some progress on this problem by geometrizing it, applying the Ehrhart theory of counting lattice points in rational polytopes (in the Beck-Zaslavsky inside-out extension) to $2q$-dimensional points that describe configurations of $q$ queens.  (This approach was suggested by Chaiken during my last visit to Albany a dozen years ago.)  We proved that the function $f_n(q)$ that counts non-attacking configurations is a quasipolynomial function of $n$ whose coefficients are polynomial functions of $q$.

However!  This leaves many open questions before one can use it to get provable formulas for $f_q(n)$.  We are now gathering in Albany to see if we can take a further leap towards a computable solution (which, to be sure, is a very distant goal).

I will describe the techniques just mentioned and some of our results, as well as the impressive computational contributions of our Czech mate Václav Kotěšovec. 
\end{abstract}

\end{document}
