\documentclass{UAmathtalk}

\author{Jim Belk}
\address{Bard College}
\urladdr{http://math.bard.edu/belk/}
\title{Turing Machines, Automata,\\ and the Brin-Thompson Group $2V$}
\date{Thursday, November 14, 2013}

\begin{document}

\maketitle

\begin{abstract}
The torsion problem for a finitely presented group asks whether a given word represents an element of finite order.  In this talk, I will present a concrete example of a group with solvable word problem but unsolvable torsion problem.  This group is known in the literature as~$2V$, and was introduced by Matt Brin in~2004 as a variant on Thompson's group~$V$.  As a consequence, I will show that there is no algorithm to determine whether a given asynchronous transducer defines a transformation of finite order.  This is joint work with Collin Bleak.
\end{abstract}

\end{document}
