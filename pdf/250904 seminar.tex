% !TeX program = xelatex
\documentclass{UAmathtalk}

\author{Matt Zaremsky}
\address{UAlbany}
\urladdr{https://www.albany.edu/~mz498674/}
\title{$\operatorname{Aut}(F_n)$ Satisfies the Boone–Higman~Conjecture}
\date{Thursday, September 4, 2025}

\renewcommand*{\where}{Humanities 124}


\begin{document}

\maketitle

\begin{abstract}
The Boone–Higman conjecture~(1973) predicts that a finitely generated group has solvable word problem if and only if it embeds in a finitely presented simple group. The “if” direction is true and easy, but the “only if” direction has been open for over 50 years. The conjecture is known to hold for various families of groups, perhaps most prominently the groups~$GL_n(\mathbb{Z})$ (due to work of Scott in~1984), and hyperbolic groups (due to work of Belk, Bleak, Matucci, and myself in~2023). In this talk I will discuss some recent work joint with Belk, Fournier-Facio, and Hyde establishing the conjecture for~$\operatorname{Aut}(F_n)$, the group of automorphisms of the free group~$F_n$, which has some surprisingly far-reaching consequences. If time permits I will also discuss some even more recent work joint with Fournier-Facio, P.~Kropholler, and Lyman, in which we find roadblocks to our finitely presented simple groups being of type~$FP_\infty$.
\end{abstract}

\end{document}
