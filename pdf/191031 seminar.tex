% !TeX program = xelatex
\documentclass{UAmathtalk}

\author{Benjamin Schweinhart}
\address{Ohio State University}
\urladdr{https://people.math.osu.edu/schweinhart.2/}
\title{Fractal Dimension Estimation\\ with Persistent Homology}
\date{Thursday, October 31, 2019}

\begin{document}

\maketitle

\begin{abstract}
Persistent homology describes the shape of a geometric object in terms of how its topology changes as it is thickened. Recently, there has been a surge of interest in applications of persistent homology including dimension estimation. We prove that the fractal dimension of a measure can be recovered from the persistent homology of random point samples, assuming the measure satisfies a standard regularity hypothesis. Our work generalizes a classical theorem of Steele on random minimum spanning trees (0-dimensional persistent homology) from the case of non-singular measures on Euclidean space to the fractal setting. We also present computational results (joint with J.~Jaquette) indicating that a persistent homology-based dimension estimation algorithm performs as well or better than classical techniques.
\end{abstract}

\end{document}
