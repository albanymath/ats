% !TeX program = xelatex
\documentclass{UAmathtalk}

\author{Marco Volpe}
\address{University of Toronto, Canada}
\urladdr{https://www.mathematics.utoronto.ca/people/directories/postdoctoral-fellows/marco-volpe}
\title{Fiberwise Simple Homotopy Theory}
\date{Thursday, April 24, 2025}

\renewcommand*{\where}{Social Sciences 256}


\begin{document}

\maketitle

\begin{abstract}
Simple homotopy theory is, roughly speaking, the study of finite CW-complexes up to collapses and expansions. From its early stages, it has been observed that simple homotopy types are deeply connected to $K$-theory. This connection is realized through Wall’s finiteness obstruction for finitely dominated complexes and the Whitehead torsion of a homotopy equivalence between finite complexes. One of Waldhausen’s main contributions (’83) to simple homotopy theory was to incorporate both Wall’s obstruction and the Whitehead torsion in the study of assembly maps in $K$-theory. Later on, Dwyer–Weiss–Williams (’03) have introduced “fiberwise” assembly maps associated to fibrations over a fixed base space, thereby providing a framework for understanding simple homotopy types varying in families.

In this talk, we introduce a novel perspective on fiberwise assembly maps, developed via the infinity-category of sheaves of spectra on a topological space. Using this approach, we are able to simultaneously generalize both the recently announced (but as yet unpublished) work of Bartels–Efimov–Nikolaus and the topological Dwyer–Weiss–Williams index theorem (’03).

This is a joint work with Maxime Ramzi and Sebastian Wolf.
\end{abstract}

\end{document}
