\documentclass{UAmathtalk}

\author{Riley Decker}
\address{UAlbany}
\title{On Attainability of $p$-Presentation Distances}
\date{Thursday, October 23, 2025}

\renewcommand*{\where}{Massry B010}

\begin{document}

\maketitle

\begin{abstract}
The $p$-presentation distance is an $\ell^p$-type generalization of the interleaving distance defined for multiparameter persistence modules and merge trees. Despite recent NP-hardness results for the computation of presentation distances, certain fundamental aspects of these distances are still poorly understood. For example, is the infimum in the definition of the \mbox{$p$-presentation} distance actually attained? By appealing to linear optimization, we answer this question in the affirmative for $p=1$ and provide bounds on the size of compatible presentations and the length of a sequence realizing the \mbox{$1$-presentation} distance. We conjecture that the infimum is attained for all $p$, and state several open problems which, if solved, would result in partial or full resolution of the conjecture. This is joint work with Mike Lesnick.
\end{abstract}

\end{document}
