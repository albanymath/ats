\documentclass[12pt]{article}

\usepackage[empty]{fullpage}
\usepackage{graphicx,helvet}

\begin{document}

\noindent\hspace{-28px}\raisebox{-19px}{%
\includegraphics[scale=0.34]{UAlogo.jpg}}%
\hfill\textsf{\textbf{\footnotesize%
Department of Mathematics and Statistics}}\bigskip\bigskip

\begin{center}\Large
  \textsf{{\huge Algebra/Topology Seminar}}\\[2.5\bigskipamount]
  \textsc{Wolfgang Steimle}\\
  {\large Bonn University, Germany}\\[\bigskipamount]
  \textsc{Obstructions to stably fibering manifolds}\\[2\bigskipamount]
  Tuesday, April 12, 2011\\ 11:45~a.m.\ in ES-146\\
  (tea \&\ coffee at 11:15 a.m.\ in ES-152)
\end{center}\bigskip\bigskip

\large\noindent\textsc{Abstract.}
Given a map $f\colon M \to B$ between compact topological manifolds, is it homotopic to the projection map of a fiber bundle whose fibers are compact manifolds? Obstructions in higher algebraic $K$-theory to fibering the given map~$f$ will be defined. The vanishing of these obstructions has a concrete geometrical meaning: the obstructions are zero if and only if $f$ fibers stably, i.e., after crossing $M$ with a high-dimensional disk. The methods also provide a classification of the different ways of stably fibering~$f$ in terms of algebraic $K$-theory.

\end{document}
