% !TeX program = xelatex
\documentclass{UAmathtalk}

\author{Brenda Johnson}
\address{Union College}
\urladdr{https://www.union.edu/mathematics/faculty-staff/brenda-johnson}
\title{What is (Functor) Calculus?}
\date{Thursday, October 12, 2023}

\renewcommand*{\where}{BB-B012}


\begin{document}

\maketitle

\begin{abstract}
Goodwillie’s calculus of homotopy functors is an important topological tool that has been used to shed light on and make connections between fundamental structures in homotopy theory and $K$-theory.  It has also inspired the creation of new types of functor calculi to tackle problems in algebra and topology.  In this talk, I will begin by describing Goodwillie’s calculus and some of these other types of functor calculi.  I will then address more general questions about what the essential features of something called “functor calculus” should be and the types of conditions and ingredients that are sufficient for creating new functor calculi.
\end{abstract}

\end{document}
