\documentclass[12pt]{article}

\usepackage{fontspec,hyperref}
\setsansfont{Helvetica Neue}

\usepackage[empty]{fullpage}

\hypersetup{pdfinfo={
     Title={Algebra/Topology Seminar, March 8, 2012},
    Author={Marco Varisco},
   Subject={Catherine Pfaff, Constructing and Classifying Fully Irreducible Outer
Automorphisms of Free Groups},
  Keywords={seminar, announcement}
}}

\begin{document}

\noindent\hspace{-28pt}\raisebox{-19pt}{\XeTeXpicfile UAlogo.jpg scaled 340}%
\hfill\textsf{\textbf{\footnotesize\href{http://www.albany.edu/math/}{Department of Mathematics and Statistics}}}\bigskip\bigskip

\begin{center}\Large
  \textsf{\huge Algebra/Topology Seminar}\\[2.5\bigskipamount]
  \textsc{Catherine Pfaff}\\
  {\large Bard College at Simon's Rock}\\[\bigskipamount]
  \textsc{Constructing and Classifying Fully Irreducible\\Outer
Automorphisms of Free Groups}\\[2\bigskipamount]
  Thursday, March 8, 2012\\ 1:15~p.m.\ in ES-143\\
  (tea \&\ coffee at 12:45 p.m.\ in ES-152)
\end{center}\bigskip\bigskip

\noindent\large\textsc{Abstract.}
The main theorem of my thesis emulates, in the context of $Out(F_r)$ theory, a mapping class group theorem (by H.~Masur and J.~Smillie) that determines precisely which index lists arise from pseudo-Anosov mapping classes.  Since the ideal Whitehead graph gives a finer invariant in the analogous setting of a fully irreducible $\phi \in Out(F_r)$, we instead focus on determining which of the 21 connected 5-vertex graphs are ideal Whitehead graphs of a geometric, fully irreducible $\phi \in Out(F_3)$.  Our main theorem accomplishes this. The methods we use for constructing fully irreducible $\phi\in Out(F_r)$, as well as our identification and decomposition techniques, can be used to extend our main theorem, as they are valid in any rank.  Our methods of proof rely primarily on Bestvina-Feighn-Handel train track theory and the theory of attracting laminations.

\end{document}
