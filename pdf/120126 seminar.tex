\documentclass[12pt]{article}

\usepackage[empty]{fullpage}
\usepackage{graphicx,helvet}

\begin{document}

\noindent\hspace{-28px}\raisebox{-19px}{%
\includegraphics[scale=0.34]{UAlogo.jpg}}%
\hfill\textsf{\textbf{\footnotesize%
Department of Mathematics and Statistics}}\bigskip\bigskip

\begin{center}\Large
  \textsf{{\huge Algebra/Topology Seminar}}\\[2.5\bigskipamount]
  \textsc{Daniel Ramras}\\
  {\large New Mexico State University}\\[\bigskipamount]
  \textsc{Decomposition Complexity of Groups\\ and Algebraic $K$-Theory}\\[2\bigskipamount]
  Thursday, January 26, 2012\\ 1:15~p.m.\ in ES-143\\
  (tea \&\ coffee at 12:45 p.m.\ in ES-152)
\end{center}\bigskip\bigskip

\large\noindent\textsc{Abstract.}
Decomposition complexity is a notion in large scale geometry that generalizes the concept of asymptotic dimension.  After introducing this concept and some of its properties, I'll explain how it can be used to study the controlled $K$-theory of metric spaces.  When applied to the family of Rips complexes of a discrete group, this leads to injectivity results for the assembly map in algebraic $K$-theory.  This is joint work with Romain Tessera and Guoliang Yu.

\end{document}
