% !TeX program = xelatex
\documentclass{UAmathtalk}

\author{Yaping Yang}
\address{University of Massachusetts Amherst}
\urladdr{http://people.math.umass.edu/~yaping/}
\title{Monodromy Representations of\\ Elliptic Braid Groups}
\date{Thursday, December 8, 2016}

\usepackage{amssymb}

\begin{document}

\maketitle

\begin{abstract}
The conformal block plays a major part in two-dimensional conformal field theory. There is a representation theoretical construction of conformal blocks attached to an affine Kac-Moody algebra and a smooth algebraic curve with marked points. The space of conformal blocks forms a vector bundle over the configuration space of points on the algebraic curve. This vector bundle carries a canonical flat connection.

In this talk (based on my joint work with Valerio Toledano Laredo), I will discuss the cases when the algebraic curve is a Riemann sphere, and an elliptic curve. The corresponding flat connections give rise to, respectively, the Knizhnik-Zamolodchikov (KZ) equation, and, by Felder-Wieczerkowski, the Knizhnik-Zamolodchikov-Bernard (KZB) equation. There are various generalizations of the KZB equations. I will focus on one generalization that was constructed by myself and Toledano Laredo recently: the elliptic Casimir connection associated to a semisimple Lie algebra~$\mathfrak{g}$. It is a holonomic system of differential equations with regular singularities on elliptic curve with marked points, taking values in a deformation of the double current algebra~$\mathfrak{g}[u, v]$ defined by Guay. The monodromy of elliptic Casimir connection leads to interesting representations of the elliptic braid groups.
\end{abstract}

\end{document}
