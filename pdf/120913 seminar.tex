\documentclass[12pt]{article}

\usepackage{fontspec,hyperref}
\setsansfont{Helvetica Neue}

\usepackage[empty]{fullpage}

\hypersetup{pdfinfo={
     Title={Algebra/Topology Seminar, September 13, 2012},
    Author={Marco Varisco},
   Subject={Ben Salisbury, Crystals of Tableaux and t-Deformation of Characters},
  Keywords={seminar, announcement}
}}

\begin{document}

\noindent\hspace{-28pt}\raisebox{-19pt}{\XeTeXpicfile UAlogo.jpg scaled 340}%
\hfill\textsf{\textbf{\footnotesize\href{http://www.albany.edu/math/}{Department of Mathematics and Statistics}}}\bigskip\bigskip

\begin{center}\Large
  \textsf{\huge Algebra/Topology Seminar}\\[2.5\bigskipamount]
  \textsc{\LARGE \href{http://www.sci.ccny.cuny.edu/~salisbury/}{Ben Salisbury}}\\
  The City College of New York\\[1.5\bigskipamount]
  \textsc{\LARGE Crystals of Tableaux and\\ t-Deformation of Characters}\\[2\bigskipamount]
  Thursday, September 13, 2012\\1:15~p.m.\ in ES-143\\[3\bigskipamount]
\end{center}

\noindent\large\textsc{Abstract.}
A combinatorial description of the crystal~$\mathcal{B}(\infty)$ for finite-dimensional simple Lie algebras in terms of certain Young tableaux was developed by J.~Hong and H.~Lee.
We establish an explicit bijection between these Young tableaux and canonical bases indexed by Lusztig's parametrization, and obtain a combinatorial rule for expressing the product side of the Gindikin-Karpelevich formula, which is a $t$-deformation of the character of~$\mathcal{B}(\infty)$, as a sum over this set of Young tableaux.
This is joint work with K.-H.~Lee.

\end{document}
