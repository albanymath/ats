\documentclass{UAmathtalk}

\author{Hara Charalambous}
\address{Aristotle University of Thessaloniki, Greece}
\urladdr{http://users.auth.gr/~hara/}
\title{Markov Bases of Lattice Ideals}
\date{Thursday, November 29, 2012}

\usepackage{amssymb}

\begin{document}

\maketitle

\begin{abstract}
One of the very basic questions in computational commutative algebra concerns the generation of ideals. Let~$R=\Bbbk[x_1,\ldots,x_n]$ where $\Bbbk$~is a field. Hilbert's basis theorem guarantees that all ideals of~$R$ are finitely generated. When $I$ is a homogeneous ideal of~$R$ then one can show that all minimal generating sets of~$I$ have the same cardinality and one can also determine the degrees that appear in all minimal generating sets.  This is not the case for nonhomogeneous ideals: they can have minimal generating sets of different cardinalities.  If $L$ is a lattice in~$\mathbb{Z}^n$ then the lattice ideal $I_L=\langle\,x^u-x^v\;\colon\;u-v\in L\,\rangle$ might not be graded and consequently its minimal generating sets might look very different. The study of these ideals is a rich subject on its own. In addition, lattice ideals have applications in diverse areas in mathematics, such as algebraic statistics, integer programming, hypergeometric differential equations, graph theory, etc. The main problem we address in this talk is how to effectively generate lattice ideals and how to determine invariants of their generating sets.
\end{abstract}

\end{document}
