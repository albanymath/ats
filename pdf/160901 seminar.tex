% !TeX program = xelatex
\documentclass{UAmathtalk}

\author{Marco Varisco}
\urladdr{http://www.albany.edu/~mv312143/}
\title{\(\mathit{TC\,}(\mathbb{A}[\Sigma_3];p)\)}
\date{Thursday, September 1, 2016}

\usepackage{amssymb}

\begin{document}

\maketitle

\begin{abstract}
In recent joint work with Wolfgang Lück, Holger Reich, and John Rognes
\href{https://arxiv.org/abs/1607.03557/}{\texttt{[arXiv:1607.03557]}},
we proved a general induction theorem for the topological cyclic homology ($\mathit{TC\,}$) of group algebras of finite groups, in the spirit of Artin and Brauer induction in the representation theory of finite groups.
The theorem states that, for any finite group~$G$, for any ring (or connective ring spectrum)~$\mathbb{A}$, and for any prime number~$p$, $\mathit{TC\,}(\mathbb{A}[G];p)$ is~determined by~$\mathit{TC\,}(\mathbb{A}[C];p)$ as $C$ ranges over the cyclic subgroups of~$G$.
Technically, we showed that the assembly map for the family of cyclic subgroups induces isomorphisms on all homotopy groups.

This result allows us to attack explicit computations for non-cyclic finite groups; more precisely, to reduce such computations to the cyclic subgroups.
A lot is known about~$\mathit{TC\,}$ of cyclic groups, but for non-abelian groups most of the previously known methods of computation do not apply.
In this talk I will describe explicitly how this works in the smallest example: that of the symmetric group~$\Sigma_3$.
\end{abstract}

\end{document}
