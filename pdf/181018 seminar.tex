% !TeX program = xelatex
\documentclass{UAmathtalk}

\author{David Meyer}
\address{Smith College}
\urladdr{https://www.davidmeyermath.com}
\title{Persistence Modules for Arbitrary Orientations of \(A_n\)}
\date{Thursday, October 18, 2018}


\begin{document}

\maketitle

\begin{abstract}
The isometry theorem says that two seemingly different metrics placed on the collection of persistence modules are equal. A persistence module which comes from data always admits the structure of a representation of a finite totally ordered set. Such posets are exactly the ones whose Hasse quiver will be the equioriented $A_n$-quiver. Since any orientation on~$A_n$ corresponds to the Hasse quiver of a finite poset, from the perspective of representation theory it's natural to wonder whether one can prove a similar theorem for persistence modules for these posets. In this talk we compare the interleaving metric with a bottleneck metric which makes use of the Auslander-Reiten quiver of the module category of the poset.
\end{abstract}

\end{document}
