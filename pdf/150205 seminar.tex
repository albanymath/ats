% !TeX program = xelatex
\documentclass{UAmathtalk}

\author{Olgur Celikbas}
\address{University of Connecticut}
\urladdr{http://www.math.uconn.edu/~olgur/}
\title{Syzygies and Tensor Product of Modules}
\date{Thursday, February 5, 2015}

\usepackage{amsmath, amsthm, amssymb}
\newtheorem*{ques}{Question}
\newcommand*{\Tor}{\operatorname{\mathsf{Tor}}}


\begin{document}

\maketitle

\begin{abstract}
Let $R$ be a (commutative and Noetherian) local hypersurface ring, i.e., $R=S/(f)$ for some regular local ring $(S, \mathfrak{n})$ with $0 \neq f\in \mathfrak{n}^2$, and let $M$ and $N$ be nonzero finitely generated $R$-modules. Assume $M$ has a rank (e.g., $R$ is a domain). Assume further $M\otimes_{R}N$ is a second syzygy module. Then a remarkable theorem of Huneke and R.~Wiegand (referred as the \emph{second rigidity theorem}) states that the pair $(M, N)$ is Tor-independent, i.e., $\Tor_{i}^{R}(M,N)=0$ for all $i\geq 1$. A consequence of this result is that both $M$ and $N$ are first syzygy modules, and $N$ is a second syzygy module.

The conclusion of the second rigidity theorem, as well as several other remarks in the literature raise the following question which is, to my knowledge, wide open in general. 

\begin{ques}
Let $R$ be a (commutative Noetherian and local) complete intersetion ring and let $M$ and $N$ be nonzero finitely generated $R$-modules. Assume $M\otimes_{R}N$ is an $n$th syzygy module for some nonnegative integer $n$. Then must $M$ or $N$ be an $n$th syzygy module? What if the pair $(M, N)$ is Tor-independent?
\end{ques}

In this talk I will discuss my joint work with Greg Piepmeyer, which gives a partial affirmative answer to this question. Our  argument makes use of a version of the New Intersection Theorem.
\end{abstract}

\end{document}
