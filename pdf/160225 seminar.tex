% !TeX program = xelatex
\documentclass{UAmathtalk}

\author{Sara Kali\v{s}nik Verov\v{s}ek}
\address{Stanford University}
\urladdr{http://web.stanford.edu/~kalisnik/}
\title{Tropical Coordinates on the Space of Persistence Barcodes}
\date{Thursday, February 25, 2016}

\usepackage{amsfonts}

\begin{document}

\maketitle

\begin{abstract}
In the last two decades applied topologists have developed numerous methods for `measuring' and building combinatorial representations of the shape of the data. The most famous example of the former is persistent homology. This adaptation of classical homology assigns a barcode, i.e., a collection of intervals with endpoints on the real line, to a finite metric space. Unfortunately, barcodes are not well-adapted for use by practitioners in machine learning tasks. I will talk about max-plus polynomials and tropical rational functions that can be used as coordinates on the space of barcodes. All of these are stable with respect to the standard distance functions (bottleneck, Wasserstein) used on the barcode space.
\end{abstract}

\end{document}
