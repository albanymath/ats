% !TeX program = xelatex
\documentclass{UAmathtalk}

\author{Liz Munch}
\urladdr{http://elizabethmunch.com/math/}
\title{The Reeb Graph Interleaving Distance Part~1}
\date{Thursday, September 3, 2015}

\begin{document}

\maketitle

\begin{abstract}
Topological data analysis (TDA) focuses on using topological techniques to summarize, study, and create insight into data. The Reeb graph is one particular construction which presents a lower-dimensional, topological signature summarizing the global structure of the connected components of levelsets of a real valued function. Since the Reeb graph is efficient to compute and is a useful descriptor for the function, it has found its place in many applications.  However, as with many other constructions in TDA, we are interested in how to deal with this construction in the context of noise.  In particular, we would like a metric to compare these structures in order to quantify their behavior, for example, across different members of a data set or with respect to noise.

In the first part of this talk, we will discuss the categorification of the Reeb graph and how this can be used to define a metric, the interleaving distance.  In the second part, we will discuss variants of the metric (such as the interleaving distance for merge trees), bounds, properties, and computational aspects.
\end{abstract}

\end{document}
