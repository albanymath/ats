% !TeX program = xelatex
\documentclass{UAmathtalk}

\author{Jose Perea}
\address{Michigan State University}
\urladdr{https://www.math.msu.edu/directory/PersonalPage.aspx?people=regularfaculty&id=111114352}
\title{Projective Coordinates\\ for the Analysis of Data}
\date{Thursday, October 8, 2015}

\begin{document}

\maketitle

\begin{abstract}
Why do we use topology to analyze data? Is it because it sounds good? Maybe\ldots\ Or is it because we have readily computable invariants? Perhaps\ldots\ But in the end, it is because we hope the invariants we compute can ultimately shed light on  the data itself. I will show in this talk how one can use persistent (co)homology, the Brown representability theorem, and other tools, to produce maps from data to the appropriate real and complex projective spaces. I will give examples of how these projective coordinates provide useful representations for certain data sets.
\end{abstract}

\end{document}
