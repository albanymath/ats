\documentclass{UAmathtalk}

\author{Jānis Lazovskis}
\address{University of Latvia}
\urladdr{https://jlazovskis.com}
\title{Deconstructing Reeb Graphs}
\date{Thursday, February 26, 2026}

\renewcommand*{\where}{Massry B012}

\begin{document}

\maketitle

\begin{abstract}
The Reeb graph of a topological space and a continuous real-valued function on it is often used to get a simplified view of the topological space, while retaining key information.
It may be considered as a 1-dimensional CW-complex, a constructible cosheaf, or a space paired with a map, and these different approaches have been useful for presenting different operations on Reeb graphs.
Recent work has developed the smoothing, truncating, and unsmoothing of Reeb graphs, and so far the relationship among them has been by measuring interleaving-type distances.
I will present ongoing work to unify and relate these different operations by finding some common definitions, from which these operations reflect the heuristic expectations of undoing and simplifying.
This is joint work with Liz Munch, Erin Chambers, and David Letscher, and is supported by the Latvian Council of Science grant 1.1.1.9/LZP/1/24/125.
\end{abstract}

\end{document}
