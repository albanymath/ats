% !TeX program = xelatex
\documentclass{UAmathtalk}

\author{Rachel Levanger}
\address{University of Pennsylvania}
\urladdr{http://rachellevanger.com}
\title{A Comparison Framework for\\ Interleaved Persistence Modules}
\date{Thursday, October 26, 2017}

\usepackage{mathtools}

\begin{document}

\maketitle

\begin{abstract}
In this talk, we'll first take a look at a recent result in the theory of persistent homology that can be used to rigorously track noise introduced during the computation of a barcode or persistence diagram. We'll then illustrate the use of this framework by looking closely at a number of examples, including common approximation techniques such as sub-sampling and discretization. In each case, we contrast this result with the typical formulation of uniform errors achieved in terms of the Bottleneck distance, which can be seen as a type of sup-norm on the space of persistence diagrams. We will also show how this framework can be used to address an open problem on non-uniform sub-level set filtrations.
\end{abstract}

\end{document}
