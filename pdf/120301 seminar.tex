\documentclass[12pt]{article}

\usepackage{fontspec,hyperref}
\setsansfont{Helvetica Neue}

\usepackage[empty]{fullpage}

\hypersetup{pdfinfo={
     Title={Algebra/Topology Seminar, March 1, 2012},
    Author={Marco Varisco},
   Subject={Brad Henry, Legendrian Knot Theory},
  Keywords={seminar, announcement}
}}

\begin{document}

\noindent\hspace{-28pt}\raisebox{-19pt}{\XeTeXpicfile UAlogo.jpg scaled 340}%
\hfill\textsf{\textbf{\footnotesize\href{http://www.albany.edu/math/}{Department of Mathematics and Statistics}}}\bigskip\bigskip

\begin{center}\Large
  \textsf{\huge Algebra/Topology Seminar}\\[2.5\bigskipamount]
  \textsc{Brad Henry}\\
  {\large Siena College}\\[\bigskipamount]
  \textsc{Legendrian Knot Theory}\\[2\bigskipamount]
  Thursday, March 1, 2012\\ 1:15~p.m.\ in ES-143\\
  (tea \&\ coffee at 12:45 p.m.\ in ES-152)
\end{center}\bigskip\bigskip

\noindent\large\textsc{Abstract.}
A contact structure on a 3-dimensional manifold refines traditional knot theory by introducing a geometric criterion for knots. Knots satisfying this criterion are called Legendrian. Since the foundational work of Bennequin in 1983, Legendrian knot theory has proven to be a fruitful area of study and a useful tool in low-dimensional topology and contact geometry. Beginning with the earliest results in the field, we will investigate classical and modern geometric techniques that have led to major advances and explore recent results that uncover deep connections between existing Legendrian knot invariants.

Floer theory, via holomorphic curve theory, has given rise to a differential graded algebra from which numerous Legendrian knot invariants are derived. Morse theory, via generating families, provides a second geometric approach to new Legendrian invariants. We will outline a program that assigns a new differential graded algebra to a Legendrian knot by studying gradient flow trees in the Morse theory of a generating family. The resulting DGA is stable-tame isomorphic to the Floer-theoretic DGA, thus providing a new connection between Floer-theoretic objects derived from Symplectic Field Theory and Morse-theoretic objects derived from generating families. This work is joint with Dan Rutherford (University of Arkansas).

\end{document}
