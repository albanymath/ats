% !TeX program = xelatex
\documentclass{UAmathtalk}

\author{Thomas Ng}
\address{Brandeis University}
\urladdr{https://sites.google.com/site/thomasng192/}
\title{Random Quotients Preserve Negative~Curvature}
\date{Thursday, September 25, 2025}

\renewcommand*{\where}{Massry B010}


\begin{document}

\maketitle

\begin{abstract}
Hyperbolic groups were introduced by Gromov in the 1980s and enjoy rich subgroup and quotient structure.  Generalizations including relative, hierarchical, and acylindrical hyperbolicity, further highlight the deep connections between algebraic properties and metric negative curvature.  I~will describe a model for constructing generic quotients of a group using independent random walks.  I will explain why such random quotients generically preserve the aforementioned aspects of negative curvature.
% 
This is joint work with C.~Abbott, D.~Berlyne, G.~Mangioni, and A.~Rasmussen.
\end{abstract}

\end{document}
