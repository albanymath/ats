% !TeX program = xelatex
\documentclass{UAmathtalk}

\author{Claudia Miller}
\address{Syracuse University}
\urladdr{https://clamille.github.io}
\title{Cyclic Adams Operations}
\date{Thursday, October 26, 2023}

\renewcommand*{\where}{BB-B012}


\begin{document}

\maketitle

\begin{abstract}
Using an idea of Atiyah from 1966, developed further in various settings by Benson, Haution, and Köck, we develop Adams operations on the Grothendieck groups of perfect complexes with support and of matrix factorizations using cyclic group actions on tensors powers. The main aim of this talk is to discuss what Adams operations are, the two traditional ways to construct them, including both lambda operations and Atiyah’s innovative approach using representations of the symmetric group to understand various power operations in algebra (given a well-behaved monoidal structure).

A second aim is to address a particular setting, time permitting: For complexes, Gillet and Soulé developed these using the first of these constructions and the Dold–Kan correspondence and used them to solve Serre’s Vanishing Conjecture in mixed characteristic (also proved independently by P.~Roberts using localized Chern characters). Their approach cannot be used in the setting of matrix factorizations, so we use Atiyah’s approach, avoiding simplicial theory altogether. We may briefly mention an application to a conjecture of Dao and Kurano on the vanishing of Hochster’s theta pairing for pairs of modules over an isolated hypersurface singularity in the remaining open case of mixed characteristic.

This is joint work with Michael Brown, Peder Thompson, and Mark Walker.
\end{abstract}

\end{document}
