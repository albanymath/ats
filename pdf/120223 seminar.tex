\documentclass[12pt]{article}

\usepackage{fontspec,hyperref}
\setsansfont{Helvetica Neue}

\usepackage[empty]{fullpage}

\hypersetup{pdfinfo={
     Title={Algebra/Topology Seminar, February 23, 2012},
    Author={Marco Varisco},
   Subject={Brad Henry, An Introduction to Contact Geometry},
  Keywords={seminar, announcement}
}}

\begin{document}

\noindent\hspace{-28pt}\raisebox{-19pt}{\XeTeXpicfile UAlogo.jpg scaled 340}%
\hfill\textsf{\textbf{\footnotesize\href{http://www.albany.edu/math/}{Department of Mathematics and Statistics}}}\bigskip\bigskip

\begin{center}\Large
  \textsf{\huge Algebra/Topology Seminar}\\[2.5\bigskipamount]
  \textsc{Brad Henry}\\
  {\large Siena College}\\[\bigskipamount]
  \textsc{An Introduction to Contact Geometry}\\[2\bigskipamount]
  Thursday, February 23, 2012\\ 1:15~p.m.\ in ES-143\\
  (tea \&\ coffee at 12:45 p.m.\ in ES-152)
\end{center}\bigskip\bigskip

\noindent\textsc{Abstract.}
A contact structure on a 3-dimensional manifold assigns to each point in the space a two-dimensional subspace of the tangent space. This 2-plane field satisfies an added non-integrability condition, which says that it is not possible to find a 2-dimensional submanifold whose tangent space agrees with the 2-plane field on an open neighborhood. In other words, the planes of the contact structure are very twisted, even locally. A 3-manifold may admit many different contact structures. The classification and investigation of these structures is commonly called contact geometry.

Although born in the work of Sophus Lie in 1896, the last three decades have been a “modern age” of renewed interest in contact geometry and its interplay with low-dimensional topology. We will discuss foundational results from the classical age of contact geometry, including stability results and the dichotomy of “tight” and “overtwisted” contact structures. From there we will venture into modern work of Giroux that utilizes convex surfaces to understand the global structure of contact structures. Given time, we will sketch Eliashberg’s contact geometric proof of Cerf’s famous theorem that any diffeomorphism of the 3-sphere extends over the 4-ball. This provided early evidence that contact geometry could be useful in tackling topological problems.

V.I. Arnold pronounced on several occasions that “contact geometry is all geometry.” I doubt I could adequately defend his remark, but I hope to convince you that contact geometry is beautiful, interesting, and useful.

\vfill\noindent\hrulefill
\begin{center}
Brad Henry will give a follow-up talk entitled \textsc{“Legendrian Knot Theory”} on March 1.
\end{center}


\end{document}
