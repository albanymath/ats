% !TeX program = xelatex
\documentclass{UAmathtalk}

\author{Lee Kennard}
\address{Syracuse University}
\urladdr{http://thecollege.syr.edu/people/faculty/pages/math/kennard-lee.html}
\title{Torus Representations with Connected Isotropy Groups and a Conjecture of Hopf}
\date{Thursday, May 2, 2019}

\begin{document}

\maketitle

\begin{abstract}
A conjecture of Hopf from the 1930s states the following: A closed, even-dimensional Riemannian manifold with positive sectional curvature has positive Euler characteristic. In joint work with Michael Wiemeler and Burkhard Wilking, this conjecture is confirmed under the additional assumption that the isometry group has rank at least five. Similar previous results required bounds on the rank that grew to infinity in the manifold dimension. The main new tool is a structural result for representations of tori with the special property that all isotropy groups are connected. Such representations are surprisingly rigid, and we analyze them using only elementary techniques. A full classification of such representations remains an open problem.
\end{abstract}

\end{document}
