% !TeX program = xelatex
\documentclass{UAmathtalk}

\author{Lucas Williams}
\address{Binghamton University, SUNY}
\urladdr{https://lucas-williams.my.canva.site}
\title{Invariants for Families of Periodic Points}
\date{Thursday, February 27, 2025}

\renewcommand*{\where}{SS-256}


\begin{document}

\maketitle

\begin{abstract}
In this talk we investigate invariants that count periodic points of a map. Given a self map~$f$ of a compact manifold we could detect $n$-periodic points of~$f$ by computing the Reidemeister trace of~$f^n$ or by computing the equivariant Fuller trace. In 2020 Malkiewich and Ponto showed that the collection of Reidemeister traces of~$f^k$ for varying~$k|n$ and the equivariant Fuller trace are equivalent as periodic point invariants, and they conjecture that for families of endomorphisms the Fuller trace will be a strictly richer invariant for $n$-periodic points. 

In this talk we will explain our new result which confirms Malkiewich and Ponto’s conjecture. We do so by proving a new Pontryagin–Thom isomorphism between equivariant parameterized cobordism and the spectrum of sections of a particular parametrized spectrum and using this result to carry out geometric computations.
\end{abstract}

\end{document}
