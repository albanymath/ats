% !TeX program = xelatex
\documentclass{UAmathtalk}

\author{Timothy Clark}
\address{Loyola University Maryland}
\urladdr{http://math.loyola.edu/~tbclark/}
\title{Combinatorial Deformations\\ of Monomial Ideals}
\date{Thursday, April 9, 2015}


\begin{document}

\maketitle

\begin{abstract}
One way to gain a structural understanding of the homological data of a monomial ideal is by \emph{deforming} it to a related ideal whose homological information (in particular its minimal free resolution) is more easily understood. A first approach in this direction was given by Bayer, Peeva, and Sturmfels, whose generic deformation produces resolutions supported on simplicial complexes. In joint work with Sonja Mapes, we use the order theoretic structure of the set of finite atomic lattices on a fixed number of atoms to describe rigid deformation, which produces resolutions supported on homology CW-posets. 
\end{abstract}

\end{document}
