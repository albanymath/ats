% !TeX program = xelatex
\documentclass{UAmathtalk}

\author{Boris Goldfarb}
\urladdr{http://www.albany.edu/~goldfarb/}
\title{The Borel Conjecture\\ Through Controlled $G$-Theory}
\date{Thursday, September 22, 2016}

\usepackage{amssymb}

\begin{document}

\maketitle

\begin{abstract}
I will survey the joint work with Gunnar Carlsson on the old conjecture of Armand Borel in topology.  The conjecture states that if a closed aspherical manifold $M$ is homotopy equivalent to another manifold then the two manifolds have to be homeomorphic.  The aspherical condition is equivalent to the universal cover of $M$ being contractible, which is common in geometry.  Our approach studies the $K$-theoretic assembly map associated to $\pi_1 (M)$ by factoring it through a controlled version of Grothendieck's $G$-theory of the group ring $\mathbb{Z} \pi_1 (M)$.  The $G$-theory turns out to be easier to compute and is equivalent to $K$-theory in very general geometric situations, for example when $\pi_1 (M)$ has finite decomposition complexity.
\end{abstract}

\end{document}
