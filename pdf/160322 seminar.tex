% !TeX program = xelatex
\documentclass[12pt]{article}

\usepackage{fontspec,hyperref}
\setsansfont{Helvetica Neue}

\usepackage[top=.66in, bottom=.5in, left=1in, right=1in]{geometry}

\hypersetup{pdfinfo={
     Title={Algebra/Topology Seminar, March 22, 2016},
    Author={Marco Varisco},
   Subject={Stefan Gille, Thomas Creutzig},
  Keywords={seminar, announcement}
}}

\begin{document}

\thispagestyle{empty}

\noindent\hspace{-28pt}\raisebox{-19pt}{\XeTeXpicfile UAlogo.jpg scaled 340}%
\hfill\textsf{\textbf{\footnotesize\href{http://www.albany.edu/math/}{Department of Mathematics and Statistics}}}\bigskip

\begin{center}\Large
  \textsf{\huge \href{http://www.albany.edu/~mv312143/seminar/}{Algebra/Topology Seminar}}\\[2.5\bigskipamount]
  \textsc{\LARGE\href{http://www.math.ualberta.ca/~gille/}{Stefan Gille}}\\
  University of Alberta, Canada\\[1.5\bigskipamount]
  \textsc{\LARGE On the Rost Nilpotence\\ Property of Threefolds}\\[1.5\bigskipamount]
  \textbf{Tuesday}, March 22, 2016, \textbf{1:10--1:50}~p.m.\ in ES-143\\[1.5\bigskipamount]
\end{center}

\noindent\large\textsc{Abstract.}
If the motive of a variety satisfies Rost nilpotence then one can ``descend'' a motivic decomposition from the one over the algebraic closure. For instance, quadrics have this property, and Rost used it to decompose the motif of a norm quadric. This result plays a crucial role in Voevodsky's proof of the Milnor conjecture. In my talk I will first discuss Rost nilpotence and then present some recent results on threefolds over fields of characteristic~0.
\bigskip

\noindent\hrulefill\bigskip
\begin{center}\Large
  \textsc{\LARGE\href{http://www.ualberta.ca/~creutzig/}{Thomas Creutzig}}\\
  University of Alberta, Canada\\[1.5\bigskipamount]
  \textsc{\LARGE Schur-Weyl Duality for\\ Heisenberg Coset Vertex Algebras}\\[1.5\bigskipamount]
  \textbf{Tuesday}, March 22, 2016, \textbf{2:00--2:40}~p.m.\ in ES-143\\[1.5\bigskipamount]
\end{center}

\noindent\large\textsc{Abstract.}
The theory of vertex algebras has many similarities with classical Lie theory. I plan to present a Schur-Weyl type duality for commuting pairs of sub vertex algebras inside a larger vertex algebra. I will also outline how this will yield new vertex algebras with potentially interesting and non semi-simple representation theory.

\end{document}
