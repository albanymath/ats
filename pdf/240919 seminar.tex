% !TeX program = xelatex
\documentclass{UAmathtalk}

\author{Håvard Bakke Bjerkevik}
\urladdr{https://www.albany.edu/math/faculty/havard-bakke-bjerkevik}
\title{Flipping Non-Crossing Spanning Trees}
\date{Thursday, September 19, 2024}

\renewcommand*{\where}{BB-B012}


\begin{document}

\maketitle

\begin{abstract}
For a set $P$ of $n$ points in general position in the plane, the flip graph $\mathcal{F}(P)$ has a vertex for each non-crossing spanning tree on $P$ and an edge between any two spanning trees that can be transformed into each other by one edge flip, i.e., the deletion and addition of exactly one edge. For $P$ in convex position, we study the diameter diam$(\mathcal{F}(P))$ of this flip graph; that is, the number of flips needed to get from a tree to another in the worst case. Modulo an additive term of size $o(n)$, a lower bound of $1.5n$ and an upper bound of $2n$ from 1999 were not improved until diam$(\mathcal{F}(P))<1.9512n$ was shown last year. We improve the lower bound to $1.\overline{5}n - O(1) = \frac{14}{9}n - O(1)$ and the upper bound to $1.\overline{6}n - 3 = \frac{15}{9}n - 3$. The lower bound disproves the conjecture that diam$(\mathcal{F}(P))\leq 1.5n$ holds for all $P$ in general position.

Joint work with Linda Kleist, Torsten Ueckerdt, and Birgit Vogtenhuber.
\end{abstract}

\end{document}
