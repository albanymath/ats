% !TeX program = xelatex
\documentclass{UAmathtalk}

\author{Daoji Huang}
\address{University of Minnesota}
\urladdr{https://www.daojihuang.me}
\title{Bumpless Pipe Dream RSK, Growth~Diagrams,~and Schubert~Structure~Constants}
\date{Thursday, November 17, 2022}

\begin{document}

\maketitle

\begin{abstract}
The cohomology ring of the complete flag variety has a basis given by classes of the Schubert varieties. A central open question in Schubert calculus is to give a combinatorial interpretation of the multiplicative structural constants of the Schubert classes. While the general question remains open, in the Grassmannian case, the Schubert structure constants are known as Littlewood–Richardson coefficients and well understood, and many of these classical rules are based on tableaux combinatorics. In this talk, we aim to generalize some of these results using bumpless pipe dreams. In particular, we introduce analogs of left and right RSK insertion for Schubert calculus of complete flag varieties. The objects being inserted are certain biwords, the insertion objects are bumpless pipe dreams, and the recording objects are decorated chains in Bruhat order. As an application, we adopt Lenart’s growth diagrams of permutations to give a combinatorial rule for Schubert structure constants in the separated descent case.
\end{abstract}

\end{document}
