\documentclass[12pt]{article}

\usepackage[empty]{fullpage}
\usepackage{graphicx,helvet}

\begin{document}

\noindent\hspace{-28px}\raisebox{-19px}{%
\includegraphics[scale=0.34]{UAlogo.jpg}}%
\hfill\textsf{\textbf{\footnotesize%
Department of Mathematics and Statistics}}\bigskip\bigskip

\begin{center}\Large
  \textsf{{\huge Algebra/Topology Seminar}}\\[2.5\bigskipamount]
  \textsc{Michael Ching}\\
  {\large Amherst College}\\[\bigskipamount]
  \textsc{Classifying Taylor Towers}\\[2\bigskipamount]
  Thursday, January 19, 2012\\ 1:15~p.m.\ in ES-143\\
  (tea \&\ coffee at 12:45 p.m.\ in ES-152)
\end{center}\bigskip\bigskip

\large\noindent\textsc{Abstract.}
I'll describe joint work with Greg Arone on a classification of Taylor towers of functors between the categories of based topological spaces and spectra. I'll try to explain how this classification is related to various categories of modules over operads and say what we know about the Taylor tower of algebraic K-theory in this context. If time permits, I'll try to indicate how a similar theory would work in a more algebraic setting.

\end{document}
