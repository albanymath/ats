% !TeX program = xelatex
\documentclass{UAmathtalk}

\author{Matt Zaremsky}
\urladdr{https://www.albany.edu/~mz498674/}
\title{Finiteness Properties of\\ Infinite Simple Groups}
\date{Thursday, September 13, 2018}


\begin{document}

\maketitle

\begin{abstract}
A group is said to be of type $F_n$ if it admits a free, cellular action on a contractible CW complex with finitely many orbits of cells up to dimension~$n$. These ``finiteness properties'' are a natural generalization of finite generation~($F_1$) and finite presentability~($F_2$). I will discuss joint work with Rachel Skipper and Stefan Witzel in which we exhibit the first known family of simple groups $G_1,G_2,\dots$ such that $G_n$ is of type $F_{n-1}$ but not~$F_n$. As a consequence we obtain the second known infinite family of quasi-isometry classes of simple groups (the first is due to Caprace and R\'emy). The groups we use arise from Nekrashevych--R\"over groups, which are mash-ups of self-similar automorphism groups of trees with Higman--Thompson groups. In this talk I will start from scratch, and not assume any familiarity with any of the above concepts.
\end{abstract}

\end{document}
