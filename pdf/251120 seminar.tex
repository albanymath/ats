\documentclass{UAmathtalk}

\author{Gabriel Angelini-Knoll}
\address{Case Western Reserve University}
\urladdr{https://www.gangeliniknoll.com}
\title{Algebraic $K$-Theory of Prime Division Rings in Homotopy Theory}
\date{Thursday, November 20, 2025}

\renewcommand*{\where}{Massry B010}

\begin{document}

\maketitle

\begin{abstract}
In algebra, the prime division rings are the finite fields with $p$ elements and the rationals. The algebraic $K$-theory groups of prime division rings are of fundamental importance. For example, algebraic $K$-theory groups of the rationals and finite fields can be used to recover the special values of the Riemann zeta function. This was originally conjectured by Lichtenbaum and Quillen in the 1970s.

In stable homotopy theory, there are more prime division rings known as Morava $K$-theory depending on a height~$n$ and a prime~$p$. Algebraic $K$-theory has been defined sufficiently generally by Waldhausen so that we can define algebraic $K$-theory of Morava $K$-theory. I will talk about some joint work with Jeremy Hahn and Dylan Wilson where we resolve a version of the conjecture of Lichtenbaum and Quillen, due to Ausoni and Rognes, for Morava $K$-theory at arbitrary heights and primes.
\end{abstract}

\end{document}

