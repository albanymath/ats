\documentclass{UAmathtalk}

\author{Raj Gandhi}
\address{Cornell University}
\urladdr{https://rajgandhi97.github.io}
\title{Motivic Segre Classes of Schubert Cells and the Connective Formal Group Law}
\date{Thursday, January 29, 2026}

\renewcommand*{\where}{Massry B012}

\begin{document}

\maketitle

\begin{abstract}
A longstanding goal of Schubert calculus is to give a positive formula for the structure constants for the Schubert basis in the cohomology ring of the $d$-step flag variety. This goal can be generalized by replacing “cohomology ring” with “torus-equivariant cohomology ring”, “$K$-ring”, and “torus-equivariant $K$-ring”. It can also be generalized in an orthogonal direction by replacing “$d$-step flag variety” with “cotangent bundle of the $d$-step flag variety”.

Recently, Allen Knutson and Paul Zinn-Justin proved a positive formula in terms of Knutson–Tao puzzles for the structure constants in the basis of motivic Segre classes of Schubert cells in (a localization of) the torus-equivariant $K$-ring of the cotangent bundle of the Grassmannian. Their proof heavily uses the theory of quantum integrable systems.

In this talk, we will describe a one-parameter deformation of the motivic Segre classes of Schubert cells in the Grassmannian which comes from the so-called “connective formal group law”, and we give a positive formula for the structure constants in the basis of deformed classes in terms of Knutson–Tao puzzles. The proof of the puzzle formula involves the representation theory of the multi-parameter quantum group of affine type~$A$.
\end{abstract}

\end{document}

