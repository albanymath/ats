% !TeX program = xelatex
\documentclass{UAmathtalk}

\author{Edward Schesler}
\address{Bielefeld University, Germany}
\urladdr{https://www.math.uni-bielefeld.de/~eschesler/}
\title{The Sigma Conjecture for Solvable S-Arithmetic Groups and Morse Theory\\on Euclidean Buildings}
\date{Thursday, November 7, 2019}

\begin{document}

\maketitle

\begin{abstract}
Given a finitely generated group $G$, the Sigma invariants of $G$ consist of geometrically defined subsets $\Sigma^k(G)$ of the set $S(G)$ of all characters $\chi\colon G \to \mathbb{R}$ of $G$. These invariants where introduced independently by Bieri-Strebel and Neumann for $k=1$ and generalized by Bieri-Renz to the general case in the late 80's in order to determine the finiteness properties of all subgroups $H$ of $G$ that contain the commutator subgroup $[G,G]$. In this talk we determine the Sigma invariants of certain $S$-arithmetic subgroups of Borel groups in Chevalley groups. In particular we will determine the finiteness properties of
every subgroup $G$ of the group of upper triangular matrices $B_n(\mathbb{Z}[1/p]) < SL_n(\mathbb{Z}[1/p])$ that contains the group $U_n(\mathbb{Z}[1/p])$ of unipotent matrices where $p$ is any sufficiently large prime number.
\end{abstract}

\end{document}
