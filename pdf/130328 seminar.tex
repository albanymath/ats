\documentclass{UAmathtalk}

\usepackage{amsfonts}

\author{Claude Schochet}
\address{Wayne State University}
\urladdr{http://www.math.wayne.edu/~claude/}
\title{Homotopy Groups of Gauge Groups}
\date{Thursday, March 28, 2013}

\renewcommand*{\abstractsize}{\normalsize}

\newcommand{\CC}{{\mathbb{C}}}
\newcommand{\QQ}{{\mathbb{Q}}}
\newcommand{\KK}{{\mathcal{K}}}

\begin{document}

\maketitle

\begin{abstract}
Let $A$ be a $C^*$-algebra. Its unitary group, $UA$, contains a wealth of topological information about $A$. However, the homotopy type of $UA$ is unknown even for $A = M_2(\CC)$.  There are various simplifications which have been considered. The first, well-traveled road, is to pass to $\pi_*(U(A\otimes \KK ))$ which is isomorphic (with a degree shift) to $K_*(A)$. This approach has led to spectacular success in many arenas, as is well-known.

A different approach is to consider $\pi _*(UA)\otimes\QQ$, the rational homotopy of $UA$, or, to be brave/reckless, to consider $\pi _*(UA)$ itself. We report on progress in the calculation of these functors for $A$ an algebra of sections of a locally trivial bundle of $C^*$-algebras over a compact metric space $X$ with $C^*$-algebra fibre $B$, so that $UA$ is the associated gauge group.  If the bundle is trivial then $UA \cong F(X, UB)$ and the Federer spectral sequence (as generalized to compact metric spaces) may be used. Our interest is the case where the bundle is non-trivial, so that $A$ is a twisted algebra. We construct a spectral sequence converging to the homotopy of the gauge group $\pi _* (UA)$ with $E_2 \cong H^*(X; \pi _* (UB))$ and a similar spectral sequence converging to $K_*(A)$.

In the case $X = S^k$ we produce a Wang sequence relating the homotopy of the gauge group $UA$ and of $UB$ and explain a conjecture identifying the differential in the gauge group sequence in terms of the classifying map of the bundle and a Samelson product.

These results are joint work and work in progress with J.~Klein, G.~Lupton, N.C.~Phillips, S.~Smith, and E.~Dror-Farjoun.
\end{abstract}

\end{document}
