% !TeX program = xelatex
\documentclass{UAmathtalk}

\author{Håvard Bakke Bjerkevik}
\urladdr{https://www.albany.edu/math/faculty/havard-bakke-bjerkevik}
\title{Simulating Time With Square-Root Space}
\date{Thursday, May 1, 2025}

\renewcommand*{\where}{Social Sciences 256}


\begin{document}

\maketitle

\begin{abstract}
I will talk about the recent paper ``Simulating Time With Square-Root Space'' (\href{https://eccc.weizmann.ac.il/report/2025/017/}{\nolinkurl{eccc.weizmann.ac.il/report/2025/017/}}) by Ryan Williams, showing that any algorithm running in time~$t(n)\geq n$ can be simulated using only roughly~$\sqrt{t\,}$ space. (Spoiler alert: an important part of the proof is a recent algorithm by Cook and Mertz (\href{https://doi.org/10.1145/3618260.3649664}{\nolinkurl{doi.org/10.1145/3618260.3649664}}), so I'll spend a lot of time discussing that paper as well.) I'll try to keep the talk accessible for a general mathy audience, and not assume any expertise in complexity theory.
\end{abstract}

\end{document}
