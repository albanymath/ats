\documentclass{UAmathtalk}

\author{Carly Briggs}
\urladdr{http://www.albany.edu/~ca488222/}
\title{An Efficient Algorithm for the Energy Function of KR-Crystals in Lie Type B}
\date{Thursday, September 12, 2013}


\begin{document}

\maketitle

\begin{abstract}
Crystals are colored directed graphs encoding information about Lie algebra representations. Certain crystals for affine Lie algebras, called Kirillov-Reshetikhin (KR) crystals, are graded by the energy function, which is defined recursively. It is desirable to calculate the energy of a vertex using only data associated with that vertex.  In recent work, C. Lenart, S. Naito, D. Sagaki, A. Schilling, and M. Shimozono gave such a formula, which is uniform across Lie types, based on the so-called quantum alcove model.  In types $A$ and $C$, this formula translates into a more efficient one, in terms of so-called Kashiwara-Nakashima columns. In this talk I will discuss partial results concerning the extension of this approach to type~$B$.
\end{abstract}

\end{document}
