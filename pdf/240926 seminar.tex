% !TeX program = xelatex
\documentclass{UAmathtalk}

\author{Catherine Pfaff}
\address{Queen’s University at Kingston, Canada}
\urladdr{https://mast.queensu.ca/~cpfaff/}
\title{What Happens When You Iterate a Free Group Automorphism \& Typical Trees in the Boundary of Outer Space}
\date{Thursday, September 26, 2024}

\renewcommand*{\where}{BB-B012}


\begin{document}

\maketitle

\begin{abstract}
As with matrices and eigenvectors, one gains important information about a free group automorphism by studying limiting objects of its repeated iteration. At the same time, while matrices act on Euclidean, spherical, and hyperbolic spaces by symmetries, automorphisms of free groups act on the “deformation space” of metric graphs by symmetries, namely Culler--Vogtmann outer space. This beautiful interplay between a space and a group of symmetries yields a deeper understanding of both the space and symmetry group. We focus on how these limiting objects, groups, and spaces communicate with each other and describe “typical” objects. Results presented are joint work with I.~Kapovich, J.~Maher, and S.J.~Taylor.
\end{abstract}

\end{document}
