\documentclass{UAmathtalk}

\author{Jim Lepowsky}
\address{Rutgers University}
\urladdr{http://www.math.rutgers.edu/people/?id=220}
\title{Logarithmic Tensor Category Theory\\ An Introduction}
\date{Thursday, November 21, 2013}

\begin{document}

\maketitle

\begin{abstract}
Tensor product operations for modules play a central role in the representation theory of many classical algebraic structures, such as Lie algebras and Hopf algebras, for example.
Module categories typically have natural rigid symmetric or braided tensor category structure, and such structure is often very elementary.
In vertex operator algebra theory, there is an analogous, natural theory, and it is far more elaborate.
The analogue of tensor category structure for modules for a semisimple Lie algebra is a ``vertex tensor category'' structure developed jointly with Yi-Zhi Huang some years ago.
More generally, the analogue for modules for a general Lie algebra is a ``logarithmic tensor category theory'' developed more recently, jointly with Huang and Lin Zhang.
I will sketch some key ideas in this work and discuss some applications and recent developments.
\end{abstract}

\end{document}
