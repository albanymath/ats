% !TeX program = xelatex
\documentclass{UAmathtalk}

\author{Boris Goldfarb}
\urladdr{http://www.albany.edu/~goldfarb/}
\title{An Example from Linear Algebra\\ Related to Topology}
\date{Thursday, September 29, 2016}

\begin{document}

\maketitle

\begin{abstract}
I want to work through a curious argument showing that the context of linear algebra is too rigid for a specific kind of computation. A better way to put it is this: there are categories of objects defined entirely in terms of free modules over a ring, but the need to compute some of their crucial invariants forces one to abandon familiar free modules and consider non-free modules. This way $K$-theory is replaced by $G$-theory in the computation. I will answer several questions.  Do we know when the Cartan map between $K$-theory and $G$-theory is an equivalence for the group rings in question?  (Yes.) Can $G$-theory use be avoided?  (No.)  My talk from last week provides topological motivation for these questions but formally is not a prerequisite for this. However, the combination of these answers gives a decent resolution of the mystery from the end of my last talk. This is a bit from joint work with Gunnar Carlsson.
\end{abstract}

\end{document}
