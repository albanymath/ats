% !TeX program = xelatex
\documentclass{UAmathtalk}

\author{David Vella}
\address{Skidmore College}
\urladdr{http://www.skidmore.edu/~dvella/dcvbio1.htm}
\title{Nilpotent Orbits for Borel Subgroups of Simple Algebraic Groups of Small Rank}
\date{Thursday, April 2, 2015}


\begin{document}

\maketitle

\vspace{-1ex}
\begin{abstract}
When a semisimple algebraic group $G$ acts on its Lie algebra $L = \mathit{Lie}(G)$ via the Adjoint representation, it is well known that there are only finitely many orbits consisting of nilpotent elements of~$L$ (called nilpotent orbits).  By classic results of Kostant, Dynkin, and others, there are nice canonical representatives of each orbit, we know the dimensions of the orbits, and we know the closure ordering on the set of such orbits for each simple algebraic group~$G$.

We attempt to mimic this program for a Borel subgroup~$B$.  That is, I would like to classify nilpotent orbits of~$B$ acting on its Lie algebra~$\mathit{Lie}(B)$ via the Adjoint representation.   Unfortunately, the results are more complicated than for semisimple groups.  For example, in 1990 Kashin proved that there always is an infinite number of nilpotent $B$-orbits, except in five small-rank cases.  In recent collaborative work with an undergraduate student, we have completely determined all the orbits in these five cases when there are only finitely many nilpotent $B$-orbits.  We give the defining equations of each orbit, compute its dimension, and determine the closure ordering in each case.

In this talk, after outlining some history and motivation why this problem is interesting to me, I will describe and summarize our results for the five finite cases.
\end{abstract}

\end{document}
