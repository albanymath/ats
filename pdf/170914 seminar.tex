% !TeX program = xelatex
\documentclass[12pt]{article}

\usepackage{fontspec,hyperref}
\setsansfont{Helvetica Neue}

\usepackage[top=.66in, bottom=.5in, left=1in, right=1in]{geometry}
\usepackage{amsfonts}

\hypersetup{pdfinfo={
     Title={Algebra/Topology Seminar, September 14, 2017},
    Author={Marco Varisco},
   Subject={Marco Varisco, Matt Zaremsky},
  Keywords={seminar, announcement}
}}

\begin{document}

\thispagestyle{empty}

\noindent\hspace{-28pt}\raisebox{-19pt}{\XeTeXpicfile UAlogo.jpg scaled 340}%
\hfill\textsf{\textbf{\footnotesize\href{http://www.albany.edu/math/}{Department of Mathematics and Statistics}}}\bigskip

\begin{center}\Large
  \textsf{\huge \href{http://www.albany.edu/~mv312143/seminar/}{Algebra/Topology Seminar}}\\[2.5\bigskipamount]
  \textsc{\LARGE Sneak Preview of the\\ \href{http://www.ams.org/meetings/sectional/2240_program.html}{AMS Sectional Meeting in Buffalo}}\\[1.5\bigskipamount]
  Thursday, September 14, 2017, 1:15--2:35~p.m.\ in ES-143\\[1.5\bigskipamount]
\end{center}
\noindent\hrulefill\bigskip

\begin{center}\Large
  \textsc{\LARGE\href{http://www.albany.edu/~mz498674/}{Matt Zaremsky}}\\[\bigskipamount]
  \textsc{\LARGE Virtual Splittings of RAAGS Over Abelian Subgroups, and Abstract Commensurability}\\[1.5\bigskipamount]
\end{center}

\noindent\small\textsc{Abstract.}
A result of Groves and Hull says that a right-angled Artin group (RAAG) splits non-trivially as an amalgamated product over an abelian subgroup if and only if it does so in an ``obvious'' way, namely if and only if its defining graph has a separating clique. Using techniques involving the Bieri-Neumann-Strebel invariant, we show that an analogous statement is even true for arbitrary finite index subgroups of RAAGs. This has consequences for the problem of classifying RAAGs up to abstract commensurability and up to quasi-isometry.
\bigskip

\noindent\hrulefill\bigskip
\begin{center}\Large
  \textsc{\LARGE\href{http://www.albany.edu/~mv312143/}{Marco Varisco}}\\[\bigskipamount]
  \textsc{\LARGE Assembly Maps for\\ Topological Cyclic Homology}\\[1.5\bigskipamount]
\end{center}

\noindent\small\textsc{Abstract.}
I will present the results of \url{http://dx.doi.org/10.1515/crelle-2017-0023}, in which we use assembly maps to study the topological cyclic homology of group algebras. Topological cyclic homology ($\mathit{TC}\,$) is a far-reaching generalization of Hochschild homology and a powerful tool in algebraic $K$-theory. We prove that, for any finite group~$G$, any connective ring spectrum~$\mathbb{A}$, and any prime~$p$, the spectrum $\mathit{TC}\,(\mathbb{A}[G];p)$ is determined by $\mathit{TC}\,(\mathbb{A}[C];p)$ as $C$ ranges over the cyclic subgroups of~$G$. More precisely, we prove that for any finite group the assembly map with respect to the family of cyclic subgroups induces isomorphisms on all homotopy groups. For infinite groups we establish pro-isomorphism, (split) injectivity, and rational injectivity results, as well as counterexamples to injectivity and surjectivity. In particular, for hyperbolic groups and for virtually finitely generated abelian groups, we show that the assembly map with respect to the family of virtually cyclic subgroups is injective but in general not surjective, in contrast to what happens in algebraic $K$-theory.

\end{document}
