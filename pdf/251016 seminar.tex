\documentclass{UAmathtalk}

\author{Conan Gillis}
\address{Cornell University}
\urladdr{https://sites.google.com/view/cgillismath}
\title{Conjugator Length, Integer Compression, and the Baumslag–Gersten Group}
\date{Thursday, October 16, 2025}

\renewcommand*{\where}{Massry B010}

\begin{document}

\maketitle

\begin{abstract}
Baumslag–Gersten's group $G=\left\langle s_0, t \mid  ts_0t^{-1}  s_0(ts_0t^{-1})^{-1}=s_0^2 \right\rangle$ is a well-known example in geometric group theory, particularly for its non-elementary-recursive (non-E.R.) Dehn function. Much work has been done to solve various decision problems in $G$, including a polynomial time algorithm for the word problem, due to Miasnikov, Ushakov, and Won, which uses a highly efficient implementation of compressed integer arithmetic based on binary sums. It is conjectured, however, that no E.R.-time algorithm exists for $G$'s conjugacy problem. To shed light on this question, we study the conjugator length function $CL(n)$ of $G$, which provides another measure of complexity of the conjugacy problem based on $G$'s intrinsic geometry. We show that, for any $\epsilon>0$, $CL(n)$ lies (up to a standard equivalence) between two power towers $2^{2^{\cdots^2}}$  of heights $ \lfloor (1-\epsilon)\log n\rfloor$ and $ \lfloor\log n\rfloor$ respectively. The talk will focus on the lower bound, where our main technique involves ``reversing'' the integer compression to obtain a lower bound on the word-length of certain elements of $G$. 
\end{abstract}

\end{document}
