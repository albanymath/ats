% !TeX program = xelatex
\documentclass{UAmathtalk}

\author{David Gepner}
\address{Purdue University}
\urladdr{http://www.math.purdue.edu/~dgepner/}
\title{On Localization Sequences\\ and Algebraic $K$-Theory}
\date{\textbf{Friday}, December 2, 2016}
\renewcommand*{\event}{Algebra/Topology Seminar --- Colloquium}%
\renewcommand*{\when}{\textbf{3:00} p.m.}%
\renewcommand*{\moreinfo}{(tea \&\ coffee at 2:30 p.m.\ in ES-152)}%

\begin{document}

\maketitle

\begin{abstract}
Localization sequences, such as the ones relating perfect complexes on an algebraic variety~$X$ with perfect complexes on an open subvariety~$U$ in~$X$ and those supported on the closed complement $Z = X - U$, play the analogous role in algebraic $K$-theory as Mayer-Vietoris squares play in ordinary cohomology. We will study certain derived analogues of these localization sequences and explain why devissage cannot hold in general for the ring objects which arise in chromatic homotopy theory.
\end{abstract}

\end{document}
