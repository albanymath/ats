\documentclass[12pt]{article}

\usepackage{fontspec,hyperref}
\setsansfont{Helvetica Neue}

\usepackage[empty]{fullpage}

\hypersetup{pdfinfo={
     Title={Algebra/Topology Seminar, March 22, 2012},
    Author={Marco Varisco},
   Subject={Richard Goldstein, Primitivity in Free Groups},
  Keywords={seminar, announcement}
}}

\begin{document}

\noindent\hspace{-28pt}\raisebox{-19pt}{\XeTeXpicfile UAlogo.jpg scaled 340}%
\hfill\textsf{\textbf{\footnotesize\href{http://www.albany.edu/math/}{Department of Mathematics and Statistics}}}\bigskip\bigskip\bigskip

\begin{center}\Large
  \textsf{\huge Algebra/Topology Seminar}\\[3\bigskipamount]
  \textsc{Richard Goldstein}\\[1.5\bigskipamount]
  \textsc{Primitivity in Free Groups}\\[3\bigskipamount]
  Part 1: Thursday, March 22, 2012, 1:15~p.m.\ in ES-143\\[\bigskipamount]
  Part 2: Thursday, March 29, 2012, 1:15~p.m.\ in ES-143
\end{center}\bigskip\bigskip\bigskip

\noindent\large\textsc{Abstract.}
A set of elements in a free group~$F$ is said to be a primitive set if it is a subset of some basis of~$F$. In these talks several theorems about primitive sets are presented. The results are applications of Whitehead's 3-dimensional model for studying automorphism of free groups; Nielsen transformations; and the folding method of labeled graphs initiated by J.~Stallings.

\end{document}
