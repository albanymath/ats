% !TeX program = xelatex
\documentclass{UAmathtalk}

\author{Michael Dougherty}
\address{Lafayette College}
\urladdr{https://mdougherty.org}
\title{Geometric Combinatorics of~Complex~Polynomials}
\date{Thursday, April 6, 2023}

\renewcommand*{\where}{BB-B008}


\begin{document}

\maketitle

\begin{abstract}
There are two commonly-used presentations for the braid group. In Artin's original presentation, we linearly order the $n$~strands and use~$n-1$ half-twists between adjacent strands to generate the group. The dual presentation, defined by Birman, Ko, and Lee in~1998, introduces additional symmetry by using the larger generating set of all half-twists between any pair of strands. Each presentation has an associated cell complex which is a classifying space for the braid group: the Salvetti complex for the standard presentation and the dual braid complex for the dual presentation. In this talk, I will present a combinatorial perspective for complex polynomials which comes from the dual presentation and describe how this leads to a cell structure for the space of complex polynomials which arises from the dual braid complex. This is joint work with Jon McCammond.
\end{abstract}

\end{document}
