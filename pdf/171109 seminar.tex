% !TeX program = xelatex
\documentclass{UAmathtalk}

\author{Justin Curry}
\urladdr{http://justinmcurry.com}
\title{Foundations of TDA:\\Classification of Constructible Cosheaves}
\date{Thursday, November 9, 2017}

\usepackage{mathtools}

\begin{document}

\maketitle

\begin{abstract}
In my last talk I presented a poset-theoretic perspective on the Elder rule, which associates a barcode (persistence diagram) to a persistent set. In this talk, I will describe a quiver and representation-theoretic perspective on persistence that emerges from a classification theorem for constructible cosheaves originally described by MacPherson and proved in various versions by Shepard, Treumann, Lurie, and myself, in collaboration with Amit Patel. This theorem is a building block for making (co)sheaves essentially finite and computable and serves as a foundation for level-set and multi-parameter persistence, the latter being the next step for TDA.
\bigskip

\noindent N.B. This is the second part of a three lecture series, but will not explicitly depend on details from the first lecture.
\end{abstract}

\end{document}
