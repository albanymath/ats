% !TeX program = xelatex
\documentclass{UAmathtalk}

\author{Eli Bashwinger}
\urladdr{https://sites.google.com/view/eli-bashwinger/}
\title{Von Neumann Algebras of~Thompson-like~Groups}
\date{Thursday, November 9, 2023}

\renewcommand*{\where}{BB-B012}
\renewcommand*{\abstractsize}{\normalsize}

\usepackage{mathrsfs}

\begin{document}

\maketitle

\begin{abstract}
Given a $d$-ary cloning system on a sequence $(G_n)_{n \in \mathbb{N}}$ of groups, we can take a ``Thompson-esque'' limit to form a Thompson-like group denoted by $\mathscr{T}_d(G_*)$, and this group canonically contains $F_d$, the smallest of the Higman--Thompson groups. The group inclusion $F_d \le \mathscr{T}_d(G_*)$ translates to an inclusion of their group von Neumann algebras $L(F_d) \subseteq L(\mathscr{T}_d(G_*))$. This talk will essentially be a survey of what is currently known about these von Neumann algebras coming from Thompson-like groups. Concerning the inclusion $L(F_d) \subseteq L(\mathscr{T}_d(G_*))$, I was able to prove it satisfies the weak asymptotic homomorphism property (WAHP), which is equivalent to $L(F_d)$ being a weakly mixing subfactor of $L(\mathscr{T}_d(G_*))$. That the inclusion satisfies the WAHP will have a number of consequences which I will discuss. Another main result is that many of these Thompson-like groups yield McDuff factors and hence are inner amenable, which is a considerable generalization of Jolissaint's result that~$L(F)$ is a McDuff factor, where $F = F_2$ is one of the ``classical" Thompson's groups.

Using cloning systems, I constructed a machine which takes in any group and produces a Thompson-like group yielding a McDuff factor. Modifying this construction, I was also able to construct another machine which takes in any finite group and any other group and produces an infinite index singular inclusion of $\mathit{II}_1$ factors without the WAHP, the first examples of their kind. Finally, using cloning systems and character rigidity, I was also able to prove the Higman--Thompson groups $F_d$ are McDuff (in the sense of Vaes--Deprez), and using the same proof I can show that, in some cases, a certain canonical subgroup of $\mathscr{T}_d(G_*)$ yields a Cartan subalgebra in $L(\mathscr{T}_d(G_*))$. 
\end{abstract}

\end{document}
